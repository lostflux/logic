\setcounter{problem}{4}

\begin{problem}
  Suppose that $\alpha$ is a wff not containing the negation symbol $\lnot$.
  Show that the length of $\alpha$ (i.e., the number of symbols in the string) is odd.

  \begin{Answer}
    We can prove that the length of $\alpha$ is odd by induction
    on the number of sentence symbols in $\alpha$.

    \textbf{\flushleft Base Case}~\label{base-case}
    
    Suppose $\alpha$ has a single sentence symbol.
    Recall that $\alpha$ may not have the negation symbol $\lnot$.
    For $\alpha$ to contain a single sentence symbol
    and be a wff, it may not have any of the binary connectives
    $\land$, $\lor$, $\rightarrow$, or $\leftrightarrow$, as those
    would necessitate the inclusion of a second sentence symbol to retain the well-formulation.
    Therefore, the wff must have a length of $1$ (since all connectives are disallowed).

    \textbf{\flushleft Inductive Step}

    Let $\alpha_n$ be a wff containing $n$ sentence symbols,
    and having arbitrary length $l$ without loss of generality.
    Consider the addition of a new sentence symbol $\beta$ to $\alpha_n$
    to yield a new wff, $\alpha_{n+1}$, containing $n+1$ sentence symbols.

    \begin{enumroman}
      \item In the simplest case, $\beta$ is adjoined to the end
        (or beginning) of $\alpha_n$
        using a connective  \\ $* \in \{ \land,\ \lor,\ \rightarrow,\ \leftrightarrow \}$,
      and the resulting phrase is wrapped in parentheses,
      resulting in the wff \\ \crim{$\alpha_{n+1}~=~(\alpha_n~*~\beta)$}
which has two additional parentheses,
      one additional connective, and one additional sentence symbol.
      Therefore, $\alpha_{n+1}$ has length $l+4$.

      \item
      In the more complicated case, $\beta$ is embedded into a subwff $\gamma$ of $\alpha_n$.
      However, the length of $\gamma$ changes by $4$ as in case (a) above.
      
    \end{enumroman}

    \noindent
    Since the length of $\alpha_1$ is known (as shown in the base case),
    we can recursively define the length of $\alpha_n$
    for any number of sentence symbols $n$.
    
    \begin{align*}
      \calL(n) = \begin{cases}
        1 \quad &\text{if $n = 1$} \\
        \calL(n-1) + 4 \quad &\text{otherwise}
      \end{cases}
    \end{align*}
    In unrolling the recursion, the length function
    becomes $\calL(n) = 4(n-1) + 1$ \\ (or~$\calL(n+1)= 4n + 1$).
    We see that:
    \begin{enumalph}
      \item $\calL(n)$ is odd for all $n \geq 1$, since we have
        the addition of $1$ to \crim{$4(n-1)$} which is always even or zero.
      \item The ratio of sentence symbols in a wff ($n+1$) to its length ($4n + 1$)
        is always greater than $1/4$, since
        \[ \frac{n + 1}{4n + 1} > \frac{n+1}{4n+4} = 1/4 \]
    \end{enumalph}
  \end{Answer}
\end{problem}

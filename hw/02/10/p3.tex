\begin{problem}
  Suppose $\fA$ and $\fB$ are two structures for $\cL$, each of which is
  countable (or possibly finite).
  Which of the following conditions imply which others?
  In each case, either explain or give a counter-example.
  \begin{enumroman}
    \item $\fA$ is isomorphic to $\fB$.
    \item $\fA$ is elementarily equivalent to $\fB$.
    \item $\equiv_\fA$ is the same relation as $\equiv_\fB$.
  \end{enumroman}
\end{problem}
\begin{Answer}
  \begin{itemize}
    \item (i) implies (ii) and (iii).\\
      If $\fA$ is isomorphic to $\fB$ under some isomorphism $\psi$,
      then $\psi(c_i^\fA) \equiv_\fB \psi(c_j^\fA)$$\fA$ whenever
      $c_i^\fA \equiv_\fA c_j^\fA$ for every constant symbol,
      and, similarly, $\psi(a_i) \equiv_\fB \psi(a_j)$ whenever
      $a_i \equiv_\fA a_j$ symbol $a_i$ in $\abs{A}$ that is not in $C$.
      Therefore, $\equiv_\fA$ and $\equiv_\fB$ are the same relation
      since they agree on all elements in the universes of the structures.
      Furthermore, whenever $\fA$ tautologically implies that
      $\alpha \equiv_\fA \beta$ for some $\alpha, \beta \in \abs{\fA}$,
      $\fB$ also tautologically implies that
      $\psi(\alpha) \equiv_\fB \psi(\beta)$, so $\fA$ and $\fB$ are
      elementarily equivalent.
    \item (ii) implies (iii), but does not necessarily imply (i).\\
      If $\fA$ is elementarily equivalent to $\fB$, then
      $c_i \equiv_\fA c_j$ if and only if $c_i \equiv_\fB c_j$,
      so $\equiv_\fA$ and $\equiv_\fB$ are the same relation.

      However, $\fA$ and $\fB$ may not be isomorphic.

      \step
      If both $\fA$ and $\fB$ have a countably infinite set of non-constant
      symbols, or they both have the same finite number of non-constant symbols,
      then they are isomorphic (since there exists a bijection between
      $\abs{\fA}$ and $\abs{\fB}$ that preserves the relation).

      \step
      If both $\fA$ and $\fB$ have a finite number of non-constant symbols
      but the number of non-constant symbols in $\fA$ does not equal the
      number of non-constant symbols in $\fB$,
      or if one of $\fA$ has an infinite number of non-constant symbols
      while the other has a finite number then there is no possible bijection
      between the sets of non-constant symbols so $\fA$ cannot be isomorphic
      to $\fB$.
    \item (iii) implies (ii) but does not necessarily imply (i)\\
      If $\equiv_\fA$ is the same relation as $\equiv_\fB$,
      then $c_i \equiv_\fA c_j$ if and only if $c_i \equiv_\fB c_j$,
      so $\fA$ and $\fB$ always agree on any two constant symbols
      being in the same
      equivalence class or being in different equivalence classes.
      Therefore, $\fA$ and $\fB$ are elementarily equivalent.

      However, as in the previous part, it is impossible for $\fA$ and $\fB$
      to be isomorphic unless they either have the same number of non-constant
      symbols or they both have a countably infinite set of non-constant symbols.

  \end{itemize}
\end{Answer}

\documentclass[9pt,reqno]{amsart}

% Include the macros file from `../common'
\input{~/latex-common/macros.tex}

\pagestyle{fancy}                       % fancy (allow headers, footers)
\fancyhf{}                              % clear all header/footer settings.
\cfoot{\thepage}                        % set page-numbers in footer.
\lhead{\textit{\textbf{ Amittai, S}}}   % set name in header, left.
\rhead{\textsc{Math 69:\@ Logic}}         % set class name in header, right.

\renewcommand{\theenumi}{\alph{enumi}}

\begin{document}
\setlength{\headheight}{14.0pt}
\setlength{\footskip}{14.0pt}

% TITLE
\reading{February 01, 2023}{Winter `23}{Marcia Groszek}{Amittai Siavava}{Math 69: Logic}


\bigskip
For this assignment, $\calL$ is the language of first-order logic with equality,
countably many constant symbols, $c_0, c_1, \ldots, c_n, \ldots$, and no other
predicate, constant, or function symbols.
We will find all the complete theories of $\calL$.
This is a single problem in five parts.
You may use completeness, soundness, and compactness.

If $\fA$ is a structure for $\calL$, define an equivalence relation on the set
$C = \set{c_0, c_1, \ldots, c_n, \ldots }$ of constant symbols of $\calL$
by \[ c_m \equiv_\fA c_n \Iff c_m^\fA = c_n^\fA, \]
That is, two constant symbols are equivalent if and only if they name the
same element of $\fA$.

Suppose that $\fA$ and $\fB$ are structures such that $\equiv_\fA$
is the same as $\equiv_\fB$.
Then two constant symbols name the same element in $\fA$ if and only if
they name the same element in $\fB$.

Let $\equiv$ be any equivalence relation on $C$.
Then there is a structure $\fA$ for $\cL$ such that $\equiv_\fA$
is the same relation as $\equiv$. Namely, let the universe of the structure
be the set of equivalence classes of constant symbols, and let each
constant symbol name its own equivalence class:
\[ \abs{\fA} = C / \equiv, \quad \text{ and } \quad \quad c_n^\fA = [c_n]. \]

The most complicated part of this is the notation.
The relation $\equiv$ on $C$ specifies whether constants $c_n$ and $c_m$
are to refer to the same element of a structure or to different elements.
Therefore, as long as $\equiv$ actually is an equivalence relation,
you can create a structure obeying those rules.

% PROBLEM 1
\newpage
\begin{problem}[5]
  Show that the formula
  \[ x = y \to P zfx \to P z fy  \]
  (where $f$ is a one-place function symbol and $P$ is a two-place predicate symbol)
  is valid.
\end{problem}
\begin{Answer}
  Suppose the formula is not valid.
  Then there exists some structure $\frakA$ and some variable assignment $s$ such that
  $\not \models_{\frakA} x = y \to P zfx \to P z fy$,
   meaning:
  \begin{align}
    \models_{\frakA} (x = y) [s] \\
    \models_{\frakA} P zfx [s]~\label{eq:5a} \\
    \not \models_{\frakA} P z fy [s] ~\label{eq:5b}
  \end{align}
  Since $\models_{\frakA} (x = y) [s]$, we have that $\sbar(x) = \sbar(y)$.

  \step
  However, since $\sbar(x) = \sbar(y)$, we have that $\sbar(fx) = \sbar(fy)$:
  \[ \sbar(fx) = f^{\frakA}(\sbar(x)) = f^{\frakA}(\sbar(y)) = \sbar(fy). \]

  \step
  Therefore, $P z fx [s]$ and $P z fy [s]$ are logically equivalent,
  so ~\ref{eq:5a} and ~\ref{eq:5b} are a contradiction,
  meaning that the formula is valid and any structure which does not satisfy
  the formula is inconsistent.
\end{Answer}


\newpage

\begin{problem}
  \emph{Recall that $\Sigma ; \alpha = \Sigma \cup \{ \alpha \}$,
  the set $\Sigma$ together with the one possibly new member $\alpha$.}
  \\
  Show that the following hold:
  \begin{enumalph}
    \item $\Sigma ; \alpha \models \beta \iff \Sigma \models (\alpha \lto \beta)$.
    \begin{Answer}
      \begin{enumroman}
        \item $\Sigma ; \alpha \models \beta \implies \Sigma \models (\alpha \lto \beta)$
        
          \step
          Suppose $\Sigma ; \alpha \models \beta$.
          Let $v$ be a truth assignment satisfying $\Sigma$.

          \step
          If $\vbar(\alpha) = T$, then $v$ satisfies $\Sigma ; \alpha$
          (since $v$ already satisfies $\Sigma$), and $\Sigma ; \alpha \models \beta$,
          implying that $\vbar(\beta) = T$.
          Therefore, $\vbar(\alpha \lto \beta) = (T \lto T) = T$.

          \step
          If $\vbar(\alpha) = F$, then $\vbar(\alpha \lto \beta) = (F \lto \vbar(\beta)) = T$.

        \step
        \item $\Sigma ; \alpha \models \beta \backimplies \Sigma \models (\alpha \lto \beta)$
        
          \step
          Suppose $\Sigma \models (\alpha \lto \beta)$ but $\Sigma ; \alpha \nmodels \beta$.

          \step
          Let $v$ be a truth assignment satisfying $\Sigma$.
          Suppose $\vbar(\alpha) = T$.
          \begin{itemize}
            \item First, we can note that
              $\vbar(\alpha) = T$ implies that $v$ satisfies $\Sigma ; \alpha$,
              which further implies that $\vbar(\beta) = F$, since $\Sigma ; \alpha \nmodels \beta$.
            \item Next, since $\Sigma \models (\alpha \to \beta)$,
              $\vbar(\alpha) = T$ implies $\vbar(\beta) = T$. This is a contradiction.
          \end{itemize}
        Therefore, it must be the case that $\Sigma \models (\alpha \lto \beta) \implies \Sigma ; \alpha \models \beta $
      \end{enumroman}
    \end{Answer}
    \item $\alpha \models \backmodels \beta \iff \models (\alpha \leftrightarrow \beta)$.  
    \begin{Answer}
      Let $v$ be any truth assignment to $\alpha$ and $\beta$
      satisfying the wff $\alpha \models \backmodels \beta$. Then:
      \begin{itemize}
        \item $\vbar(\beta) = T$ whenever $\vbar(\alpha) = T$ (since $\alpha \models \beta)$.
        \item $\vbar(\alpha) = T$ whenever $\vbar(\beta) = T$ (since $\beta \models \alpha)$.
        \item Consequently, $\lnot (\vbar(\alpha)) \leftrightarrow \lnot (\vbar(\beta))$,
        implying that $\vbar(\alpha \leftrightarrow \beta) = T$.
      \end{itemize}
      Therefore, any truth assignment to $\alpha$ and $\beta$
      satisfying the wff $\alpha \models \backmodels \beta$ also satisfies
      $\models (\alpha \leftrightarrow \beta)$.
    \end{Answer}
  \end{enumalph}
\end{problem}


\newpage
% \setcounter{problem}
\begin{problem}[5]
  Let $X$ be the set of all wffs of sentential logic
  and $\equiv$ be tautological equivalence.
  Define a binary (2-place) relation on equivalence classes by
  \[ [\alpha] \models [\beta] \iff \alpha \models \beta. \]
  Determine whether this relation is well-defined
  and prove your answer is correct.
\end{problem}


\newpage
\begin{problem}
  Write formulas that define the following sets in $\fA$.  

  \step
  Be sure your formulas have the correct free variables.
  Also be sure to notice that this language has only two
  nonlogical symbols other than $\forall$, namely $0$ and $<$.
  
  \step
  This is a short answer problem; you do not need to prove
  that your formulas define the sets they are supposed to define.

  \begin{enumalph}
    \item $\{-2\}$.
      \begin{Answer}

      \end{Answer}
    \item $\{0 , -2 \}$.
      \begin{Answer}

      \end{Answer}
    \item $\{ (n,m) \mid m = n +  1\}$.
      \begin{Answer}

      \end{Answer}
    \item $\{n \mid n \geq -2\}$.
      \begin{Answer}

      \end{Answer}
  \end{enumalph}
\end{problem}


\newpage
\begin{problem}
  We define the difference of two sets $X$ and $Y$,
  written $X - Y$, to be the set of all members of $X$
  that are not members of $Y$.
  Suppose $X$ is an effectively enumerable set of expressions
  and $Y$ is a decidable set of expressions.
  \begin{enumalph}
    \item Show that $X - Y$ is effectively enumerable.
      \begin{Answer}
        Since $X$ is effectively enumerable, we can write a program to list
        all the elements of $X$, although the program may never halt.
        On the other hand, $Y$ is decidable, so given any wff $\alpha$,
        we can determine whether $\alpha$ is a member of $Y$.

        \step
        To determine the members of $X - Y$, we can proceed as follows:
        \begin{enumroman}
          \item List a member of $X$ that has not yet been listed.
          \item Check whether the listed wff is a member of $Y$.
            \begin{itemize}
              \item If it is a member of $Y$, discard the wff.
              \item If it is not a member of $Y$, list it as a member of $X - Y$.
            \end{itemize}
          \item Go to step (i).
        \end{enumroman}
        Since all members of $X - Y$ are members of $X$,
        every such wff will be listed as a member of $X$,
        therefore $X - Y$ is effectively enumerable.
      \end{Answer}
    \item Suppose that $X$ is not decidable, and $X \subseteq Y$,
      show that $Y - X$ is not effectively enumerable.
      \begin{Answer}
        Since all decidable sets are effectively enumerable,
        we can effectively list all the members of $Y$,
        although an algorithm to do so may never halt.

        \step
        However, to determine that a member of $Y$ is a member of $Y - X$,
        we must determine that they are not a member of $X$,
        and $X$ is not decidable so we cannot do so. 
        Therefore, we cannot determine whether listed members of $Y$ are members of $Y - X$,
        and $Y - X$ is not effectively enumerable.
      \end{Answer}
  \end{enumalph}
\end{problem}


\vfill

\end{document}

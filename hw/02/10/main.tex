\documentclass[9pt,reqno]{amsart}

% Include the macros file from `../common'
% ENCODING
\usepackage[utf8]{inputenc}
\usepackage{pmboxdraw}
% \usepackage{pmboxdraw-extras}

% text alignment
\usepackage{array, ragged2e}

% URLs
\usepackage{hyperref}

% Sets
\newcommand{\FF}{\mathbb{F}}
\newcommand{\NN}{\mathbb{N}}
\newcommand{\QQ}{\mathbb{Q}}
\newcommand{\RR}{\mathbb{R}}
\newcommand{\ZZ}{\mathbb{Z}}

\newcommand{\F}{\mathbb{F}}
\newcommand{\N}{\mathbb{N}}
\newcommand{\Q}{\mathbb{Q}}
\newcommand{\R}{\mathbb{R}}
\newcommand{\Z}{\mathbb{Z}}
\newcommand{\C}{\mathbb{C}}
\renewcommand{\b}{\{0,1\}}

% SET THEORY
\newcommand{\nequiv}{\not\equiv}
\renewcommand{\notin}{\not\in}

% FUNCTIONS
\usepackage{amsmath}
\DeclareMathOperator{\cost}{cost}
\DeclareMathOperator{\len}{len}
\DeclareMathOperator{\rank}{rank}
\DeclareMathOperator{\sgn}{sgn}
\DeclareMathOperator{\wt}{wt}

% ALGEBRA
\newcommand{\GL}{\mathrm{GL}}
\DeclareMathOperator{\id}{id}
\DeclareMathOperator{\M}{M}
\DeclareMathOperator{\SL}{SL}
\DeclareMathOperator{\Syl}{Syl}
\DeclareMathOperator{\tr}{tr}
\DeclareMathOperator{\ev}{ev}

% Combinatorics

% Probability
\newcommand{\Hd}{\texttt{\color{BrickRed}H}}
\newcommand{\Tl}{\texttt{T}}
\newcommand{\EE}{\mathop{\mathbb{E}}}
\newcommand{\PP}{\mathop{\mathbb{P}}}
\newcommand{\indic}{\mathbb{1}}
\DeclareMathOperator{\Var}{Var}

% End of proof marker
\newcommand{\qedblack}{\hfill\ensuremath{\blacksquare}}
\newcommand{\qedwhite}{\hfill\ensuremath{\square}}

% Calligraphic caps
\newcommand{\calA}{\mathcal{A}}
\newcommand{\calB}{\mathcal{B}}
\newcommand{\calC}{\mathcal{C}}
\newcommand{\calD}{\mathcal{D}}
\newcommand{\calE}{\mathcal{E}}
\newcommand{\calF}{\mathcal{F}}
\newcommand{\calI}{\mathcal{I}}
\newcommand{\calL}{\mathcal{L}}
\newcommand{\calO}{\mathcal{O}}
\newcommand{\calP}{\mathcal{P}}
\newcommand{\calS}{\mathcal{S}}
\newcommand{\calT}{\mathcal{T}}
\newcommand{\calU}{\mathcal{U}}
\newcommand{\calV}{\mathcal{V}}
\newcommand{\calX}{\mathcal{X}}
\newcommand{\calY}{\mathcal{Y}}
\newcommand{\calZ}{\mathcal{Z}}

% Boldface letters
\newcommand{\ba}{\mathbf{a}}
\newcommand{\bb}{\mathbf{b}}
\newcommand{\bp}{\mathbf{p}}
\newcommand{\bq}{\mathbf{q}}
\newcommand{\br}{\mathbf{r}}
\newcommand{\bu}{\mathbf{u}}
\newcommand{\bv}{\mathbf{v}}
\newcommand{\bx}{\mathbf{x}}
\newcommand{\by}{\mathbf{y}}
\newcommand{\bz}{\mathbf{z}}
\newcommand{\tbp}{\mathbf{\widetilde{p}}}

% Special math symbols
\newcommand{\eps}{\varepsilon}
\newcommand{\ceq}{\subseteq}
\newcommand{\ang}[1]{\langle{} #1 \rangle}
\newcommand{\ceil}[1]{\lceil{} #1 \rceil}
\newcommand{\floor}[1]{\lfloor{} #1 \rfloor}

% Problem names and other small-caps constants
\newcommand{\inv}{\textsc{inv}\xspace}

% Useful for marking steps of a derivation to explain later
\newcommand{\circled}[1]{\raisebox{.5pt}{\textcircled{\raisebox{-.1pt}{\scriptsize #1}}}}


% Page size and margins
% \usepackage[left=1in,right=1in,top=1.3in,bottom=1.3in,nofoot]{geometry}
\usepackage{fancyhdr}   % for fancy header
\usepackage{fancyvrb}   % for fancy verbatim
\usepackage{graphicx}   % for including images
\usepackage{enumerate}  % for enumerating lists
\usepackage{enumitem}
\usepackage[rgb, dvipsnames]{xcolor}
\usepackage{tcolorbox}

\usepackage{multicol}

% Answer BOX
\usepackage{microtype}
\usepackage{mdframed}
\newmdenv[%
  skipabove=6pt,
  skipbelow=6pt,
  innertopmargin=6pt,
  leftmargin=-5pt,
  rightmargin=-5pt, 
  innerleftmargin=5pt,
  innerrightmargin=5pt,
  backgroundcolor=black!10
]{Answer}%


% Header BOX
\newcommand{\handout}[6]{
  \noindent
  \begin{center}
  \setlength{\fboxrule}{1.2pt}
  \framebox{
    \vbox{
      \hbox to 5.78in { \textbf{#6} \hfill {\bf #2} }
      \vspace{4mm}
      \hbox to 5.78in { {\Large \hfill {\textbf{ #5 }}  \hfill} }
      \vspace{2mm}
      \hbox to 5.78in { {\textit{\textbf{#3 \hfill #4}}} }
    }
  }
  \setlength{\fboxrule}{0.2pt}
  \end{center}
  \vspace*{4mm}
}

% Header BOX
\newcommand{\PSET}[5]{\handout{#1}{#2}{Prof.\ #3}{Student: #4}{PSET #1}{#5}}

\newcommand{\homework}[5]{\handout{#1}{#2}{Prof.\ #3}{Student: #4}{Homework assigned #1}{#5}}

\newcommand{\reading}[5]{\handout{#1}{#2}{Prof.\ #3}{Student: #4}{Reading assigned #1}{#5}}

% Credit Statement
\newcommand{\CreditStatement}[1]{
  \noindent
  \begin{center} {
    \bf Credit Statement
  }
  \end{center}
  { #1 }
}

% Problem Counter
\newenvironment{problem}[1][]%
{%
\stepcounter{problem}
\setcounter{section}{\value{problem}}
\vspace{.2cm} \noindent {\center  \textbf{\\ Problem \arabic{problem}.\\}} {\noindent}~%
}{%
% \vspace{.2cm}%
}
% Package Imports
\usepackage{amssymb,amsthm,amsmath,amstext}
\usepackage{mathdots} % for \dots
  % \dotsc -- dots with commas.
  % \dotsb -- dots with binary operators.
  % \dotsm -- multiplication dots.
  % \dotsi -- dots with integrals.
  % \dotso -- "other dots".
\usepackage{wasysym, stackengine, makebox, tikz-cd}
\newcommand\isom{\mathrel{\stackon[-0.1ex]{\makebox*{\scalebox{1.08}{\AC}}{=\hfill\llap{=}}}{{\AC}}}}
\newcommand\nvisom{\rotatebox[origin=cc] {-90}{$ \isom $}}
\newcommand\visom{\rotatebox[origin=cc] {90} {$ \isom $}}


% Custom colors
\definecolor{crimson}{rgb}{0.86, 0.08, 0.24}
\definecolor{teal}{rgb}{0.0, 0.5, 0.5}
\definecolor{zaffre}{rgb}{0.0, 0.08, 0.66}
\definecolor{DarkOliveGreen}{rgb}{0.33, 0.42, 0.18}
\newcommand{\crim}{\textcolor{crimson}}
\newcommand{\teal}{\textcolor{teal}}
\newcommand{\zaff}{\textcolor{zaffre}}
\newcommand{\black}{\textcolor{black}}
\newcommand{\darkgreen}{\textcolor{DarkOliveGreen}}
\newcommand{\green}{\textcolor{OliveGreen}}

% Block coloring
\newenvironment{blockcolor}{\par \color{crimson} {\par}}

% \newcommand{\id}{\mathbf{id}\;}

% matrices -- vertical separators
\makeatletter
\renewcommand*\env@matrix[1][*\c@MaxMatrixCols{ c}]{%
  \hskip -\arraycolsep{}
  \let\@ifnextchar\new@ifnextchar{}
  \array{#1}}
\makeatother

% \usepackage{accode}
\usepackage{tikz}

% long multiplications
\usepackage{xlop}

% custom functions.
\renewcommand{\gcd}[2]{\mathbf{gcd}\;(#1,\;#2)}
\newcommand{\lcm}[2]{\mathbf{lcm}\;(#1,\;#2)}
\newcommand{\Therefore}{\dot{.\hspace{.095in}.}\hspace{.095in}}
\newcommand{\However}{\dot{}\hspace{.045in}.\hspace{.045in} \dot{}\hspace{.095in}}

% resume includes
\usepackage[utf8]{inputenc}
\usepackage[full]{textcomp}
\usepackage{CJKutf8}
\usepackage[lf]{ebgaramond}

\usepackage[OT1]{fontenc}
\usepackage{enumitem}
\usepackage[scale=.75]{geometry}
\usepackage{url}

\pagestyle{headings}

\setlength\parindent{2em}

\thispagestyle{empty}

\newcommand{\cvsubsection}[1]{\subsection*{\hspace{1.45em}#1}}

\usepackage{enumitem}

% enumalph
\newenvironment{enumalph}{
  \begin{enumerate}[label=(\alph*)]
}{\end{enumerate}}

% enumroman
\newenvironment{enumroman}{
  \begin{enumerate}[label=(\roman*)]
}{\end{enumerate}}

\newenvironment{problab}[1]
{\noindent\textbf{Problem #1}.}
{\vskip 6pt}
\theoremstyle{remark}
\newtheorem*{solu}{Solution}

% import := definition
\usepackage{colonequals}

\usepackage{listings}
\usepackage{color}

% -- Defining colors:
\usepackage[dvipsnames]{xcolor}
\definecolor{codegreen}{rgb}{0,0.6,0}
\definecolor{codegray}{rgb}{0.5,0.5,0.5}
\definecolor{codepurple}{rgb}{0.58,0,0.82}
\definecolor{backcolour}{rgb}{0.95,0.95,0.92}
\definecolor{dkgreen}{rgb}{0,0.6,0}
\definecolor{gray}{rgb}{0.5,0.5,0.5}
\definecolor{mauve}{rgb}{0.58,0,0.82}

\lstset{frame=tb,
  backgroundcolor=\color{backcolour},   
  commentstyle=\color{codepurple},
  keywordstyle=\color{NavyBlue},
  numberstyle=\tiny\color{codegray},
  stringstyle=\color{codepurple},
  basicstyle=\ttfamily\footnotesize\bfseries,
  breakatwhitespace=false,         
  breaklines=true,                 
  captionpos=t,                    
  keepspaces=true,                 
  numbers=left,                    
  numbersep=5pt,                  
  showspaces=false,                
  showstringspaces=false,
  showtabs=false,                  
  tabsize=2,
  % escapeinside={\%*}{*)},          % if you want to add LaTeX within your code
}

\usepackage{booktabs}


\DeclareMathOperator{\Aut}{Aut}
\DeclareMathOperator{\opspan}{span}
\DeclareMathOperator{\Tr}{Tr}
\DeclareMathOperator{\Frac}{Frac}
\DeclareMathOperator{\ord}{ord}
\DeclareMathOperator{\Sym}{Sym}

\numberwithin{equation}{section}
\newtheorem{theorem}[equation]{Theorem}
\newtheorem{thm}[equation]{Theorem}
\newtheorem{lemma}[equation]{Lemma}
\newtheorem{lem}[equation]{Lemma}
\newtheorem{proposition}[equation]{Proposition}
\newtheorem{prop}[equation]{Proposition}
\newtheorem{corollary}[equation]{Corollary}
\newtheorem{cor}[equation]{Corollary}

\theoremstyle{definition}
\newtheorem{definition}[equation]{Definition}
\newtheorem{defn}[equation]{Definition}
\newtheorem{example}[equation]{Example}
\newtheorem{xca}[equation]{Exercise}
\newtheorem{notation}[equation]{Notation}
\theoremstyle{remark}
\newtheorem{remark}[equation]{Remark}

\numberwithin{equation}{section}


\usepackage[normalem]{ulem}
\usepackage{fullpage}
\usepackage{colonequals}
\usepackage{amssymb}
\usepackage{amsthm}
\usepackage{amsmath}
\usepackage{amsxtra}
\usepackage{mathtools}
\usepackage{mathrsfs}

\usepackage{hyperref}
\hypersetup{colorlinks=true,urlcolor=blue,citecolor=blue,linkcolor=blue}

\numberwithin{equation}{section}

\usepackage{amssymb}
\usepackage{amsthm}
\usepackage{amsmath}
\usepackage{amsxtra}

\setlength{\hfuzz}{4pt}

\DeclarePairedDelimiter{\abs}{\lvert}{\rvert}

\newcommand{\defi}[1]{\textsf{#1}} % for defined terms

\renewcommand{\baselinestretch}{1.5} 

\usepackage{titling}
\usepackage[english]{babel}
\usepackage[utf8]{inputenc}
\usepackage{amsmath, amsfonts, amsthm}
\usepackage{graphicx}
\usepackage[colorinlistoftodos]{todonotes}
\usepackage{subfig}
% \usepackage{mdframed} 
\usepackage{color}
\usepackage{tabu}
\usepackage{tikz}
\usepackage{enumerate}
\usepackage{multicol}
\usepackage{pgfplots}
\usepackage{csquotes}
\pgfplotsset{compat=1.18}
\usepackage[style=iso]{datetime2}
\usepackage{multirow}
\usepackage{float}  % prevent table repositioning.
\usepackage{lipsum}


% SQL Symbols
\def\ojoin{\setbox0=\hbox{$\bowtie$}%
  \rule[-.02ex]{.25em}{.4pt}\llap{\rule[\ht0]{.25em}{.4pt}}}
\def\leftouterjoin{\mathbin{\ojoin\mkern-5.8mu\bowtie}}
\def\rightouterjoin{\mathbin{\bowtie\mkern-5.8mu\ojoin}}
\def\fullouterjoin{\mathbin{\ojoin\mkern-5.8mu\bowtie\mkern-5.8mu\ojoin}}

\newcounter{problem}
\setcounter{problem}{0}

\renewcommand{\theenumi}{\alph{enumi}}

\usepackage{MnSymbol}
\def \backmodels{\leftmodels}


\usepackage[english]{cleveref}

\def \vbar{\overline{v}}
\DeclareRobustCommand\iff{\;\Longleftrightarrow\;}
\DeclareRobustCommand\lto{\rightarrow}
\DeclareRobustCommand\backimplies{\Longleftarrow}

\DeclareRobustCommand{\step}{\vspace{0.5em}\noindent}

\DeclareRobustCommand{\set}[1]{\left\{#1\right\}}


\pagestyle{fancy}                       % fancy (allow headers, footers)
\fancyhf{}                              % clear all header/footer settings.
\cfoot{\thepage}                        % set page-numbers in footer.
\lhead{\textit{\textbf{ Amittai, S}}}   % set name in header, left.
\rhead{\textsc{Math 69:\@ Logic}}         % set class name in header, right.

\renewcommand{\theenumi}{\alph{enumi}}

\begin{document}
\setlength{\headheight}{14.0pt}
\setlength{\footskip}{14.0pt}
\title{
  \Huge{Math 69: Logic}\\
  Abelian Groups
}
\date{\Large{\today}}

\begin{titlingpage}

  \author{\Large{Amittai Siavava}}
  \maketitle

  \step
  \begin{center}
    \textbf{Abstract}
  \end{center}
  \step
  \begin{quote}
    \small
    This paper describes a set of axioms and a language for abelian groups
    and explores what the theory of abelian groups can tell us
    about completeness.

    \step
    \crim{
      I will be expanding the abstract once I have worked mroe on the conclusions
      and have a better understanding of the ultimate shape/direction of the paper.
    }
  \end{quote}
\end{titlingpage}


\bigskip
For this assignment, $\calL$ is the language of first-order logic with equality,
countably many constant symbols, $c_0, c_1, \ldots, c_n, \ldots$, and no other
predicate, constant, or function symbols.
We will find all the complete theories of $\calL$.
This is a single problem in five parts.
You may use completeness, soundness, and compactness.

If $\fA$ is a structure for $\calL$, define an equivalence relation on the set
$C = \set{c_0, c_1, \ldots, c_n, \ldots }$ of constant symbols of $\calL$
by \[ c_m \equiv_\fA c_n \Iff c_m^\fA = c_n^\fA, \]
That is, two constant symbols are equivalent if and only if they name the
same element of $\fA$.

Suppose that $\fA$ and $\fB$ are structures such that $\equiv_\fA$
is the same as $\equiv_\fB$.
Then two constant symbols name the same element in $\fA$ if and only if
they name the same element in $\fB$.

Let $\equiv$ be any equivalence relation on $C$.
Then there is a structure $\fA$ for $\cL$ such that $\equiv_\fA$
is the same relation as $\equiv$. Namely, let the universe of the structure
be the set of equivalence classes of constant symbols, and let each
constant symbol name its own equivalence class:
\[ \abs{\fA} = C / \equiv, \quad \text{ and } \quad \quad c_n^\fA = [c_n]. \]

The most complicated part of this is the notation.
The relation $\equiv$ on $C$ specifies whether constants $c_n$ and $c_m$
are to refer to the same element of a structure or to different elements.
Therefore, as long as $\equiv$ actually is an equivalence relation,
you can create a structure obeying those rules.

% PROBLEM 1
\newpage
\begin{problem}
  Show that if $\fA$ is any finite structure for $\cL$, there is a countable
  structure $\fB$ such that $\fB$ is elementarily equivalent to $\fA$,
  and in $\fB$, infinitely many elements are not named by constant symbols.
  In other words, we have that
  \[ \set{b \in \abs{\fB} \mid \forall n (b \ne c_n^\fB)} \] is infinite.
  (\emph{Hint: } Use compactness.)

  \crim{
    \textbf{Compactness Theorem: } If $\Gamma \models \phi$,
    then for some finite $\Gamma_0 \subseteq \Gamma$,
    $\Gamma_0 \models \phi$.
  (In other words, a set $\Gamma$ has a model iff every finite subset
  has a model)\\
  \textbf{Elementary equivalence: } Two structures $\fA$ and $\fB$ are
  \emph{elementarily equivalent} if
  $\quad \models_\fA \alpha \Iff \models_\fB \alpha$,
}
\end{problem}
\begin{Answer}
  Let $\fA$ be a \emph{finite} structure for $\cL$
  such that $\abs{\fA} \ne \infty$.

  \step
  Define $\fB$ as such
  \begin{enumroman}
    \item Include translations of the constant symbols from
      $C$ in $\abs{\fB}$. That is; for each $c_n \in C$,
      $c_n^\fB \in \abs{\fB}$.\\
      Whenever $c_m^\fA = c_n^\fA$, let $c_m^\fB = c_n^\fB$
      so that  $c_m \equiv_\fB c_n$ if and only if $c_m =_\fA c_n$.
    \item Define a new set of countably many elements
      $B = \set{b_0, b_1, b_2, \ldots, b_n, \ldots}$ such that:
      \begin{itemize}
        \item $\forall i \forall j \lnot (b_i = c_j)$.
        \item $\forall i \forall j (i = j \lor \lnot (b_i = b_j))$.
      \end{itemize}
    \item For each $b_i \in B$, let $b_i^\fB \in \abs{\fB}$.
  \end{enumroman}

  \step
  Then $\fB$ is an infinite structure for $\cL$ that is elementarily
  equivalent to $\fA$.
\end{Answer}


\newpage
\begin{problem}
  Let $X$ be the set of all wffs of sentential logic
  and $\equiv$ be tautological equivalence.
  Define a binary ($2$-place) function on equivalence classes,
  which we could call conjunction, by
  \[ [\alpha ] \bigland [\beta ] = [\alpha \land \beta ] \]
Prove that this function is well-defined.

\step
\emph{
  As you do this, at some point you are going to have to prove that two
  wffs are tautologically equivalent.
}
  
\step
\emph{
  For this exercise, please do this by showing
  explicitly that any truth assignment that satisfies one of the formulas also
  satisfies the other, and conversely.
}

\step
\emph{
  You may think it’s obvious that these wffs are tautologically equivalent.
  I agree, and after this proof, you can get away with saying so, or giving a
  more informal explanation, in similar circumstances.
}
\end{problem}


\newpage
\begin{problem}
  With the same definitions as in problem (2),
  let $\frakB$ be the \emph{reduct} of $\frakB^*$ to a structure
  for $\calL$. This means that:
  \begin{align*}
    \abs{\fB} &= \abs{\fB^*} \\
    f^{\fB} &= f^{\fB^*} \\
    P^\fB &= P^{\fB^*}
  \end{align*}

  \step
  In other words, $\fB$ translates every parameter $\calL$
  in exactly the same way as $\fB^*$.
  It merely has no translation for $E$, since $E$ is not a symbol of
  $\calL$. (Of course, $\fB$ must translate $=$ as equality).

  \step
  Because $\fB^*$ translated $E$ as equality, we can conclude that
  for any wff $\alpha$ of $\calL$ and variable assignment $\sbar$
  for $\fB$, since $\alpha^*$ is $\alpha$ with $E$ replaced by $=$,
  \[ \fB \models \alpha [\sbar ] \Iff \fB^* \models \alpha^*[\sbar]. \]
  This makes sense because $\fB$ and $\fB^*$ have the same universe,
  so $\sbar$ is also a variable assignment for $\fB$.

  \step
  By problem (2),
  \[ \fB^* \models \alpha [h \circ s] \Iff \fA^* \models \alpha^* [s], \]
  and therefore
  \[ \fB \models \alpha [h \circ s] \Iff \fA^* \models \alpha^* [s]. \]

  \step
  We conclude that if $\Sigma$ is any set of formulas of $\calL$
  including all the logical axioms in groups $5$ and $6$,
  and the set $\Sigma^*$ (obtained by replacing $=$ with $E$
  in every element of $\Sigma$) is satisfiable
  by some structure $\fA^*$ and variable assignment $s$,
  then the original set $\Sigma$ is also satisfiable by the structure
  $\fB$ and variable assignment $h \circ s$ as defined in problem (2).

  \step
  Using problem (1) and the further context, show that if every consistent
  set of sentences in $\calL^*$ is satisfiable then every consistent set of
  sentences in $\calL$ is satisfiable.
  
  \begin{Answer}
    We shall prove the contrapositive:
    suppose some consistent set of sentences in $\calL$
    is not satisfiable. Then there is some set $\Sigma$ of sentences
    in $\calL^*$ that must not be satisfiable.

    \step
    Let $\gamma$ be a variable in $\calL$ that is not satisfied under an
    assignment $s$, then $\fA^* \models \lnot a [s]$.

    \step
    Therefore, $\fB \models \lnot a [h \circ s]$.
    By the compactness theorem, there exists some finite subset
    of $\fB_0 \subseteq \abs{\fB}$ such that $\fB_0 \deduces \gamma$.
    Then either $\fB_0$ contains $\lnot \gamma$ or $\fB_0$ contains
    or deduces some formulae $\alpha$ and $\beta$
    such that $\alpha$ = is $\beta \lto \gamma$.
    so $\fB$ is also not satisfiable.
  \end{Answer}
\end{problem}


\newpage
\begin{problem}
  Define a relation $\equiv$ on $(\N^+)^2 = \set{(x, y) : x, y \in \N^+}$ by
  \[ (x_1, y_1) \equiv (x_2, y_2) \Iff x_1y_2 = x_2y_1. \]
  \begin{enumalph}
    \item Show that $\equiv$ is an equivalence relation on $(\N^+)^2$.
      \begin{Answer}
        \begin{enumroman}
          \item \textbf{Reflexivity:}
            By definition, $(x_1, y_1) \equiv (x_2, y_2)$ if and only if $x_1 y_2 = x_2y_1$.
            For any arbitrary element $\alpha = (x, y)$, we always have that $x y = x y$
            so $\alpha \equiv \alpha$.
          \item \textbf{Symmetry:}
            Let $\alpha = (x_1, y_1)$ and $\beta = (x_2, y_2)$
            such that $\alpha \equiv \beta$, then $x_1 y_2 = x_2 y_1$.
            Since moving the left-hand side of the equation to the right-hand side
            and the right-hand side to the left-hand side does not change the equality,
            we also have that $x_2 y_1 = x_1 y_2$ so $\beta \equiv \alpha$.
          \item \textbf{Transitivity:}
            Let $a = (x_1, y_1)$, $b = (x_2, y_2)$, and $c = (x_3, y_3)$
            such that $a \equiv b$ and $b \equiv c$.
            Then $x_1 y_2 = x_2 y_1$ and $x_2 y_3 = x_3 y_2$.
            Therefore;
            \begin{align*}
              a \equiv b \Iff x_1 y_2 = x_2 y_1, \quad \text{so } \quad &\frac{x_1 y_2}{x_2 y_1} = 1 \\
              b \equiv c \Iff x_2 y_3 = x_3 y_2, \quad \text{ so } \quad &\frac{x_2 y_3}{x_3 y_2} = 1 \\
              &\Therefore \frac{x_1 y_2}{x_2 y_1} \cdot \frac{x_2 y_3}{x_3 y_2} = 1 \\
              &\Therefore \frac{x_1 \crim{y_2} \zaff{x_2} y_3}{\zaff{x_2} y_1 x_3 \crim{y_2}} = 1\\
              &\Therefore \frac{x_1 y_3}{x_3 y_1} = 1 \\
              &\Therefore x_1 y_3 = x_3 y_1 \\
              &\Therefore a \equiv c
            \end{align*}
        \end{enumroman}
      \end{Answer}
    \newpage
    \item Describe the equivalence class of $(3, 3)$
      (do this without mentioning the equivalence relation $\equiv$).
      \begin{Answer}
        The equivalence class of $(3, 3)$ is the set of all pairs $(x, y)$
        such that $3x = 3y$.
        This is the set of all pairs $(x, y)$ such that $x = y$.
        \[ \set{(x, y) : x, y \in \N \text{ and } x = y}\]
      \end{Answer} 
    \item Suppose that we try to define a function on equivalence classes by
      \[ f[(x, y)] = [(2x^2, 2y^2)]. \]
      Either show that this function is well-defined or show that it is not.
      \begin{Answer}
        This function is well-defined.
        Let $a = (x_1, y_1)$ and $b = (x_2, y_2)$ be two elements chosen from the same
        equivalence class without loss of generality,
        then $x_1 y_2 = x_2 y_1$.

        \step
        Let $a' = f[a] = [(2x_1^2, 2y_1^2)]$ and $b' = f[b] = [(2x_2^2, 2y_2^2)]$.

        \step
        First, we show that $a' \equiv b'$, since $(2x_1^2) \cdot (2y_2^2) = (2x_2^2) \cdot (2y_1^2)$.
        \begin{align*}
          2x_1^2 \cdot 2y_2^2 &= 4x_1^2 y_2^2 \\
          &= (2x_1y_2)^2 \\
          &= (2x_2y_1)^2 \quad \quad \text{ since $x_1y_2 = x_2 y_1$}\\
          &= 4x_2^2 y_1^2 \\
          &= 2x_2^2 \cdot 2y_1^2
        \end{align*}

        \step
        Therefore, whenever two elements $a$ and $b$ are in the same equivalence class,
        we also have that $f[a] \equiv f[b]$.

      \end{Answer}
  \end{enumalph}
\end{problem}


\newpage
\begin{problem}
  We define the difference of two sets $X$ and $Y$,
  written $X - Y$, to be the set of all members of $X$
  that are not members of $Y$.
  Suppose $X$ is an effectively enumerable set of expressions
  and $Y$ is a decidable set of expressions.
  \begin{enumalph}
    \item Show that $X - Y$ is effectively enumerable.
      \begin{Answer}
        Since $X$ is effectively enumerable, we can write a program to list
        all the elements of $X$, although the program may never halt.
        On the other hand, $Y$ is decidable, so given any wff $\alpha$,
        we can determine whether $\alpha$ is a member of $Y$.

        \step
        To determine the members of $X - Y$, we can proceed as follows:
        \begin{enumroman}
          \item Start the algorithm to list members of $X$.
          \item Each moment a member is listed, check whether the listed wff
             is a member of $Y$ (since $Y$ is decidable).
            \begin{itemize}
              \item If it is a member of $Y$, discard the wff.
              \item If it is not a member of $Y$, list it as a member of $X - Y$.
            \end{itemize}
          \item Repeat from step (i).
        \end{enumroman}
        Since all members of $X - Y$ are members of $X$,
        every such wff will be listed as a member of $X$,
        therefore $X - Y$ is effectively enumerable.
      \end{Answer}
    \newpage
    \item Suppose that $X$ is not decidable, and $X \subseteq Y$,
      show that $Y - X$ is not effectively enumerable.
      \begin{Answer}
        Since all decidable sets are effectively enumerable,
        we can effectively list all the members of $Y$,
        although an program to do so may never halt.

        \step
        We could have proceeded as above,
        only swapping the roles of $X$ and $Y$.
        However, to determine that a member of $Y$ is a member of $Y - X$,
        we must determine that they are not a member of $X$,
        and $X$ is not decidable so we cannot do so.
        Therefore, even though we can list members of $Y$,
        we cannot determine the membership of all such members in $Y - X$,
        so $Y - X$ is not effectively enumerable.

        \step
        Note: the fact that both $X$ and $Y$ are effectively enumerable
        does not help us in this case.
        To effectively list all the members of $Y - X$,
        we would have to somehow list all the members of $Y$,
        list all the members of $X$, then compare the two.
        However, the programs to list all the members of $Y$ and $X$
        are not guaranteed to halt, so we cannot compare the two.

      \end{Answer}
  \end{enumalph}
\end{problem}


\vfill

\end{document}

\begin{problem}
  Let $\calL$ be a language for first-order logic, including the equality symbol
  (and any number of other symbols),
  and $\calL^*$ be $\calL$ without the equality symbol,
  but with an additional $2$-place predicate symbol $E$.
  For any formula $\alpha$ of $\calL$, define $\alpha^*$ to be the formula
  of $\calL^*$ obtained from$\alpha$ by replacing every occurrence of
  $=$ with $E$. If $\Sigma$ is a set of formulas of $\calL$,
  define \[ \Sigma^* = \set{ a^* \mid a \in \Sigma }. \]

  \begin{enumalph}
    \item This is a short answer problem.\\
      For which logical axiom groups is it true that $\alpha$ is a logical
      axiom in that group if and only if $\alpha^*$ is a logical axiom in
      that group?
      \begin{Answer}
        Groups 5 and 6.
      %   \begin{enumroman}
      %     \item Group 1: True
      %     \item Group 2: True
      %     \item Group 3: True
      %     \item Group 4: True
      %     \item Group 5: True
      %     \item Group 6: True
      %   \end{enumroman}
      \end{Answer}
      
    \item This is a short answer problem.\\
      Which, if any, of the following are true?
      \begin{enumroman}
        \item If $\alpha_1, \alpha_2, \ldots, \alpha_n$
          is a deduction from $\Sigma$ in the language $\calL$
          then $\alpha^*_1, \alpha^*_2, \ldots, \alpha^*_n$
          is a deduction from $\Sigma^*$ in the language $\calL^*$
          \begin{Answer}
            True
          \end{Answer}
        \item If $\alpha^*_1, \alpha^*_2, \ldots, \alpha^*_n$
          is a deduction from $\Sigma^*$ in the language $\calL^*$
          then $\alpha_1, \alpha_2, \ldots, \alpha_n$
          is a deduction from $\Sigma$ in the language $\calL$
          \begin{Answer}
            False
          \end{Answer}
      \end{enumroman}
    \item Show that if $\Sigma$ is a consistent set of formulas in $\calL$
      then $\Sigma^*$ is a consistent set of formulas in $\calL^*$.\\
      You may use your answers in (a) and (b).\\
      \emph{Hint: } Recall that $\Sigma$ is inconsistent if there is
      some $\alpha$ such that $\Sigma \deduces \alpha$ and
      $\Sigma \deduces \lnot \alpha$.
      \begin{Answer}
        Suppose $\Sigma$ is consistent and $s$ is a variable assignment
        that satisfies $\Sigma$, then $\sbar(\sigma) = T$ for all $\sigma \in \Sigma$.
        Then there exists some deduction
        $\vector{\alpha_1, \alpha_2, \ldots, \alpha_n}$ with $\alpha_n = \alpha$
        and each $\alpha_i$ either occurring in $\Sigma$ or having some
        $j, k < i$ such that $\Sigma \models \alpha_j$ and $\alpha_k$ is
        $\alpha_j \lto \alpha_i$.
        Therefore, the deduction $\vector{\alpha_1^*, \alpha_2^*, \ldots, \alpha_n^*}$
        also exists in $\Sigma^*$ with $\alpha_n^* = \alpha^*$,
        meaning $\Sigma^* \models \alpha^*$.

        \step
        Therefore, if $\Sigma$ is consistent and some truth assignment satisfies
        all rules in $\Sigma$, then $\Sigma^*$ is also consistent.
      \end{Answer}

    \item Show by example that it is possible for $\Sigma$ to be inconsistent
      but $\Sigma^*$ to be consistent.
      \begin{Answer}
        Suppose both $\cL$ and $\cL^*$ contain two constant symbols $c_1$
        and $c_2$ defined such that $c_1 \ne c_2$, and a variable symbol $x$.
        Suppose $\Sigma$ contains the two sentences $a = c_2$ and $a = c_2$,
        then $\Sigma$ is not consistent since it deduces two contradicting
        results. For instance, if $\alpha$ is the assertion that $a = c_1$
        then $\Sigma \models \alpha$ and $\Sigma \models \lnot \alpha$.
        
        \step
        On the other hand, if we change equality in $\cL$ with a predicate
        symbol $E$ in $\cL^*$,
        then $\Sigma^*$ contains the two sentences
        $E a_i c_1$ and $E a_i c_2$. Thus, $\Sigma \models E a_i c_1$
        and $\Sigma \models E a_i c_2$. However, there is no rule
        in $\Sigma^*$ that says $\forall a (E a c_1 \lto \lnot E a c_2)$,
        so $\Sigma^*$ is still consistent.
      \end{Answer}
  \end{enumalph}

\end{problem}

\begin{problem}[8]
  Assume the language (with equality) has just the parameters
  $\forall$ and $P$, where $P$ is a two-place predicate symbol.
  Let $\fA$ be the structure with $\abs{\fA} = \Z$,
  the set of the integers (positive, negative, zero)
  and with $\vector{a, b} \in P^\fA$ iff $\abs{a - b}~=~1$.

  \step
  Thus, $\fA$ looks like an infinite graph:
  \begin{align*}
    \cdots
    \longleftrightarrow \bullet
    \longleftrightarrow \bullet
    \longleftrightarrow \bullet
    \longleftrightarrow \cdots
  \end{align*}
  Show that there is an elementarily equivalent structure $\fB$ that is not
  connected (being \emph{connected} means that for every two members of
  $\abs{\fB}$, there is a path between them. A path --- of length $n$ ---
  is a sequence $\vector{p_0, p_1, \ldots, p_n}$ of elements of $\abs{\fB}$
  such that $a = p_0$ and $b = p_n$ and $\vec{p_i, p_{i+1}} \in P^\fB$
  for each $i$).

  \step
  Write down sentences saying $c$ and $d$ are far apart,
  applying compactness.

\end{problem}

\begin{Answer}
  Let $\pi$ be a new constant symbol, then 
  we define $\fB$ to be the structure with $\abs{\fB} = \Z \cup \set{\pi}$.
  Next, we specify that $\pi$ does not belong to $P^\fB$
  by adding a sentence to the language saying:
  \[ \forall p \; \lnot P^\fB \; \pi p. \]
  Then $\fB$ is not connected, since the new constant $\pi$
  does not have a path of length $1$ to any other element of $\abs{\fB}$,
  and the existence of a path of higher length depends on the existence
  of a path of a lower length.
  We can specify this formally with an extra set of sentences saying that
  there does not exist any path of any length from a point $p$ to the constant
  $\pi$:
  \[ 
    \set{
      \forall d \forall p_1 \forall p_2 \ldots \forall p_n
      \lnot ((P^\fB d p_1) \land (P^\fB \; p_1 p_2) \land (P^\fB \; p_2 p_3) 
      \land \ldots \land (P^\fB \; p_{n-1} p_n) \land (P^\fB \; p_n \pi))
      \mid d \in \abs{\fB}, n \in \Z
    }
  \]

  \step
  However, $\fB$ is elementarily equivalent to $\fA$
  since whenever $\fA \models P^\fA p_i p_j$ then
  $\fB \models P^\fB p_i p_j$ and vice versa. 
\end{Answer}

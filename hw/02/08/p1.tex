\begin{problem}[2]
  Prove the equivalence of parts (a) and (b) of the completeness theorem.

  \step
  \emph{Suggestion: } $\Gamma \models \phi$ iff $\Gamma \cup \set{\lnot \phi}$
    is unsatisfiable. And $\Delta$ is satisfiable iff $\Delta \ne \bot$,
    where $\bot$ is some unsatisfiable, refutable formula
    like $\lnot \forall x x = x$.
    (Similarly, the soundness theorem is equivalent to the statement
    that every satisfiable set of formulas is consistent.)
    \begin{blockcolor}
      \textbf{Completeness Theorem: }
      \begin{enumalph}
        \item If $\Gamma \models \phi$, then $\Gamma \deduces \phi$.
        \item Any consistent set of formulas is satisfiable.
      \end{enumalph}
    \end{blockcolor}
\end{problem}

\begin{Answer}
  We shall prove this in two parts: (i) that (a) implies (b),
  and (ii) that (b) implies (a).

  \bigskip
  Suppose $\Gamma$ is a set of formulas such that (a) holds,
  then we claim that $\Gamma$ is consistent if and only if it is satisfiable.

  \step
  Assume $\Gamma$ is not satisfiable, then there exists some formula
  $\alpha$ such that $\Gamma \models \alpha$ and $\Gamma \models \lnot \alpha$.
  By part (a) of the completeness theorem, this implies that
  $\Gamma \deduces \alpha$ and $\Gamma \deduces \lnot \alpha$.
  But deductions are finite, so there must exist some finite subset
  $\Gamma_0 \subseteq \Gamma$ such that $\Gamma_0 \deduces \alpha$,
  and there must exist some finite subset $\Gamma_1 \subseteq \Gamma$
  such that $\Gamma_1 \deduces \lnot \alpha$.
  Therefore:
  \begin{itemize}
    \item Either $\Gamma_0$ deduces a sentence that says $\alpha$,
      or $\Gamma_0$ or deduces two sentences $\gamma_1$ and $\gamma_2$
      where $\gamma_1$ says $\gamma_2 \lto \alpha$.
    \item Either $\Gamma_1$ deduces a sentence that says $\lnot \alpha$,
      or $\Gamma_0$ deduces two sentences $\gamma_3$ and $\gamma_4$
      where $\gamma_3$ says $\gamma_4 \lto \lnot \alpha$.
  \end{itemize}
    Since $\Gamma \supset \Gamma_0 \cup \Gamma_1$, $\Gamma$
      deduces each of $\gamma_1, \gamma_2, \gamma_3$, and $\gamma_4$, so
      $\Gamma$ deduces both $\alpha$ and $\lnot \alpha$, so it must not
      be consistent.

  \step
  \crim{Therefore, given that if $\Gamma \models \phi$ then $\Gamma \deduces \phi$,
    for $\Gamma$ to be consistent it must be satisfiable.}

  \bigskip
  Next, suppose $\Gamma$ is a consistent set of wffs and (b) holds ---
  that is, $\Gamma$ is consistent and satisfiable.
  We show that whenever $\Gamma \models \phi$, then $\Gamma \deduces \phi$.

  \step
  Let $\phi$ be a wff and $\Gamma \models \phi$.
  Then $\Gamma \cup \set{\lnot \phi}$ is unsatisfiable,
  meaning there exists a deduction of $\phi$ from
  $\Gamma \cup \set{\lnot \phi}$,
  and there also exists a deduction of $\lnot \phi$ from
  $\Gamma \cup \set{\lnot \phi}$.
  Deductions are finite, so there must exist some finite subset
  $\Gamma_0 \subseteq \Gamma \cup \set{ \lnot \phi }$ so that
  $\Gamma_0 \deduces \phi$ and there must exist some finite subseteq
  $\Gamma_1 \subseteq \Gamma \cup \set{ \lnot \phi }$ so that
  $\Gamma_1 \deduces \lnot \phi$.
  
  But since the original set $\Gamma$ is consistent
  (therefore has no contradicting deductions):
  It must be the addition of $\lnot \phi$ to $\Gamma$
  that causes the contradiction. But $\set{\lnot \phi}$ deduces
  $\lnot \phi$, so $\Gamma$ must have had no deduction of $\lnot \phi$.
  Additionally, $\lnot \phi \notin \Gamma_0$, since
  $\Gamma_0 \deduces \phi$ and $\Gamma_0 \not \deduces \lnot \phi$.
  Since $\Gamma_0 \subseteq \Gamma \cup \set{ \lnot \phi }$
  and $\lnot \phi \notin \Gamma_0$, $\Gamma_0 \subseteq \Gamma$.
  Therefore, $\Gamma \deduces \phi$ (since its subset deduces $\phi$).

  \crim{ Therefore, given that any consistent set of formulas is satisfiable,
    if $\Gamma \models \phi$ then $\Gamma \deduces \phi$.}

\end{Answer}

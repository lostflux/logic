\begin{problem}[9]
  Suppose that $X$ is a set and $\le$ is a preordering of $X$.
  Define a new binary relation on $X$ by
  \[ x \equiv y \iff (x \le y \land  y \le x). \]
  Show that $\equiv$ is an equivalence relation on $X$,
  that $\le$ induces a well-defined relation on equivalence classes,
  and that this induced relation is a partial ordering of $X/\equiv$.
  
\end{problem}
\begin{Answer}
  \begin{claim}
    $\equiv$ is an equivalence relation on $X$.

    \begin{proof}
      We need to show that $\equiv$ is reflexive, transitive, and symmetric.

      \step
      \textbf{Reflexivity:} $x \le x$ for all $x \in X$, so $x \equiv x$,
        so the equivalence relation is reflexive.
      
      \step
      \textbf{Transitivity:} For $x, y, z \in X$, suppose $x \equiv y$ and $y \equiv z$, then:
      \begin{align}
        x \equiv y &\iff (x \le y \land y \le x)~\label{eq:9.1} \\
        y \equiv z &\iff (y \le z \land z \le y)~\label{eq:9.2} \\
        ~\ref{eq:9.1} \land ~\ref{eq:9.2} &\iff (x \le z \land z \le x) \iff (x \equiv z)~\nonumber
      \end{align}
      \step
      \textbf{Symmetry:} For $x, y \in X$, suppose $x \equiv y$, then:
      \begin{align*}
        x \equiv y &\iff (x \le y \land y \le x) \\
        &\iff (y \le x \land x \le y) \\
        &\iff (y \equiv x)
      \end{align*}
    \end{proof}
  \end{claim}

  \newpage
  \begin{claim}~\label{claim:9.2}
    $\le$ induces a well-defined relation on equivalence classes.
    \begin{proof}
      Let $x, y \in X$ such that $x \equiv y$.
      Take any $z \in X$ without loss of generality.
      If $z \equiv x$, then $z \equiv y$ since $x \equiv y$.
      On the other hand, suppose $z \nequiv y$. Then it may not be the case that
      $z \equiv x$, as that would imply that $x \equiv y$ (since $z \equiv x$ and $x \equiv y$).
      This implies that, for any $z \in X$, either
      (1) \green{$z \equiv x$, and $z \equiv x_i$ for all $x_i \equiv x$},
      or (2) \crim{$z \nequiv x$, and $z \nequiv x_i$ for all $x_i \equiv x$}.
      Therefore, $\le$ induces a well-defined relation on equivalence classes.
    \end{proof}
  \end{claim}
  \begin{claim}
    The induced relation is a partial ordering of $X/\equiv$.
    \begin{proof}
      We need to show that $\equiv$ is reflexive, transitive, and antisymmetric
      when applied to equivalence classes of $x$.

      \step
      \textbf{Reflexivity:} Suppose $[x]$ and $[y]$ are equivalence classes on $X$.
      Suppose $[x] \equiv [y]$.
      Then $x_i \equiv y_j$ for all $x_i \in [x]$ and $y_j \in [y]$.
      Since $\equiv$ is symmetric when applied to members of $X$,
      $y_j \equiv x_i$ for all $x_i \in [x]$ and $y_j \in [y]$,
      so $[x] \equiv [y]$ implies $[y] \equiv [x]$.

      \step
      \textbf{Transitivity:} Suppose $[x] \equiv [y]$ and $[y] \equiv [z]$.
      Then $x_i \equiv y_j$ for all $x_i \in [x]$ and $y_j \in [y]$,
      and $y_j \equiv z_k$ for all $y_j \in [y]$ and $z_k \in [z]$,
      since $\equiv$ is transitive when applied to members of $X$.
      Therefore, $x_i \equiv z_k$ for all $x_i \in [x]$ and $z_k \in [z]$,
      so $[x] \equiv [z]$.

      \step
      \textbf{Antisymmetry:} Suppose $[x] \equiv [y]$ and $[y] \equiv [x]$.
      As we saw in the proof to claim ~\ref{claim:9.2},
      this implies that every $x_i \in [x]$ is equivalent to every $y_j \in [y]$,
      so everything in $[x]$ is in the equivalence class
      of everything in $[y]$, which is only possible if $[x] = [y]$.
    \end{proof}
  \end{claim}
\end{Answer}

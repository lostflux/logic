\begin{problem}
  Prove Theorem 17F:
  \emph{
    A set of expressions is decidable iff
    both it and its complement 
    (relative to the set of all expressions)
    are effectively enumerable.
  }

  \step
  Remark: Two semidecision procedures make a whole.
\end{problem}
\begin{Answer}
  Let $\Sigma$ be a set of expressions,
  with $\Gamma$ as its complement.

  \step
  We first prove that if $\Sigma$ is decidable
  then both $\Sigma$ and $\Gamma$
  are effectively enumerable:

  \noindent
  Suppose $\Sigma$ is decidable, then we can always determine
  whether a wff $\alpha$ is in $\Sigma$ or is not in $\Sigma$.
  Therefore, we can implement an enumeration algorithm
  as follows:
  \begin{enumroman}
    \item Pick an arbitrary wff, $\beta$, that has not yet been listed as a member of $\Sigma$ or $\beta$.
    \item If $\beta \in \Sigma$, then list it as an element of $\Sigma$.
    \item However, if $\beta \not\in \Sigma$, then list it as an element of $\Gamma$.
    \item Repeat from step $1$.
  \end{enumroman}
  
  \bigskip
  \step
  We then prove that if both $\Sigma$ and its complement,
  $\Gamma$, are effectively enumerable then $\Sigma$ is decidable.
  
  \noindent
  Assume that both $\Sigma$ and $\Gamma$ are effectively enumerable.
  Then, by definition of effectively enumeration,
  we may list members of $\Sigma$ and non-members of $\Sigma$
  (i.e. members of $\Gamma$),
  and every member or non-member will eventually be listed in the appropriate category
  even if the enumeration might never end in the case of an infinite $\Sigma$ or $\Gamma$.
  Consequently, by checking the listed wffs we may always determine
  whether a wff $\beta$ is in $\Sigma$ or not in $\Sigma$, implying that $\Sigma$ is decidable.
\end{Answer}


\setcounter{problem}{4}
\begin{problem}
  Translate each English sentence into the first-order language specified.
  \emph{(
    You may want to carry out the translation in several steps,
    as in some of the examples.) Make full use of the notational
    conventions and abbreviations to make the end result
    as readable as possible.}

    \step
    Symbols:

    \step
    \begin{enumroman}
      \item $\forall$, for all things;
      \item $P$, is a person;
      \item $T$, is a time;
      \item $Fxy$, you can fool x at y.
    \end{enumroman}
    
    \step
    One or more of the above may be ambiguous,
    in which case you will need more than one translation.
  
  \begin{enumalph}
    \item You can fool some of the people all of the time.
      \begin{Answer}
        There exists a person you can fool all the time.
        Equivalently, it is not the case that for all people,
        you cannot fool them all the time.
        \begin{align*}
          \lnot \forall x (\lnot \forall t ((Px \land Ty) \lto Fxy))
        \end{align*}
      \end{Answer}
    \item You can fool all of the people some of the time.
      \begin{Answer}
        There exists a time when you can fool all of the people.
        Equivalently, it is not the case that for all times,
        there exists a person that you cannot fool.
        \begin{align*}
          \lnot \forall t (\lnot \forall x ((Px \land Ty) \lto Fxy))
        \end{align*}
      \end{Answer}
    \item You can’t fool all of the people all of the time.
      \begin{Answer}
        There exists a time when you cannot fool all of the people.
        Equivalently, it is not the case that for all times,
        you can fool all of the people.
        \begin{align*}
          \lnot \forall t (\forall x ((Px \land Ty) \lto Fxy))
        \end{align*}

      \end{Answer}
  \end{enumalph}
\end{problem}

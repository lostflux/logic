\begin{problem}
  Show that tautological equivalence is an equivalence relation
  on the set of wffs of sentential logic;
  that is, if we define
  \[ \alpha \equiv \beta \iff \alpha \models \backmodels \beta, \]
  then $\equiv$ is an equivalence relation on the set of wffs.
\end{problem}
\begin{Answer}
  \begin{enumroman}
    \item Show that $\equiv$ is reflexive.
    
    \step
    Let $\alpha$ be a under some truth assignment $v$.

    \step
    If $\vbar(\alpha) = T$, then $\alpha \models \alpha$.
    We also have that $\alpha \backmodels \alpha$ by the same argument.

    \step
    If $\vbar(\alpha) = F$, then $\alpha \models \models \alpha$
    since we may conclude anything from a false statement.
    The same argument applies to $\alpha \backmodels \alpha$
    since the deduction itself is the assignment to $\alpha$,
    and $\alpha$ has been assigned to $F$.
    \emph{
      If we were deducing a different variable, say $\gamma$,
      then if $\vbar(\alpha) = F$, and $\vbar(\gamma) = T$,
      we would have $\vbar(\alpha \models \gamma) = T$
      but $\vbar(\gamma \models \alpha) = F$,
      so $\vbar(\alpha \models \backmodels \gamma) = F$
      and $\alpha \nequiv \gamma$.
    }


    \item
      Show that $\equiv$ is symmetric.

      \step
      Let $\alpha \equiv \beta$ under some truth assignment $v$.
      By definition, $\alpha \models \backmodels \beta$.
      This implies that:
      \begin{itemize}
        \item if $\vbar(\alpha) = T$, then $\vbar(\beta) = T$;
        \item if $\vbar(\alpha) = F$, then $\vbar(\beta) \ne T$,~\label{ref:tautological-equivalence:symmetric}
          as that would imply $\vbar(\alpha \backmodels \beta) = F$.
          Therefore, $\vbar(\beta) = F$.
      \end{itemize}
      Therefore, $\vbar(\alpha) = \vbar(\beta)$, so $\vbar(\beta \models \backmodels \alpha) = T$, and $\beta \equiv \alpha$.
    \item
      Show that $\equiv$ is transitive.

      \step
      Let $\alpha \equiv \beta$ and $\beta \equiv \gamma$ under some truth assignment $v$.
      By definition, $\alpha \models \backmodels \beta$ and $\beta \models \backmodels \gamma$.
      As shown in part \ref{ref:tautological-equivalence:symmetric},
      If $\vbar(x \models \backmodels y) = T$, then $\vbar(x) = \vbar(y)$.
      Therefore, $\vbar(\alpha) = \vbar(\beta)$ and $\vbar(\beta) = \vbar(\gamma)$,
      implying that $\vbar(\alpha) = \vbar(\gamma)$.
      Therefore, $\alpha \models \backmodels \gamma$, so $\alpha \equiv \gamma$.
  \end{enumroman}
\end{Answer}

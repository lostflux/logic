\begin{problem}
  Let $\calL$ be the language for first-order logic with two-place
  predicate symbols $E$ and $P$ and one-place function symbol $f$.
  (We are not assuming that $\calL$ has the equality symbol.
  On the other hand, we are not ruling out the possibility that
  $\calL$ has the equality symbol and/or any number of parameter symbols
  in addition to $\forall$, $E$, $P$, and $f$.
  Other symbols are not relevant to this question.)

  \step
  Suppose $\frakA$ is a structure for $\calL$ that is a model of the sentence
  \[ \forall xExx \]
  and of every sentence of the form
  \[
    \forall x \forall y \forall z_1 \forall z_2 \ldots \forall z_n
    (Exy \lto (\alpha \lto \alpha')).
  \]
  where $\alpha$ is an atomic formula with variables included among
  $\set{x,y,z_1,z_2,\ldots,z_n}$, and $\alpha'$ is
  obtained from $\alpha$ by replacing some
  (possibly none, possibly some but not all, possibly all)
  occurrences of $x$ by $y$.
  Examples of sentences of this form are
  \[ \forall x \forall y (Exy \lto (Exx \lto Eyx))
  \text{ and }
  \forall x \forall y \forall z (Exy \lto (Ez fx \lto Ez fy)).
  \]
  An example of a sentence \emph{not} of this form is
  \[
    \forall x \forall y \forall z (Exy \lto (Ez fy \lto Ez fx)).
  \]

  \step
  Show that $E^\frakA$ is an equivalence relation on the universe $\abs{\frakA}$,
  that $P^\frakA$ induces a well-defined relation on equivalence classes, and that $f^\frakA$
  induces a well-defined function on equivalence classes.
\end{problem}
\begin{Answer}
  Let $s$ be a variable assignment satisfying the conditions as stated.
  Then:
  \step
  \begin{enumalph}
    \item $E^\frakA$ is an equivalence relation on the universe $\abs{\frakA}$.
      \begin{enumroman}
        \item $E$ is reflexive: This follows from the sentence $\forall x Exx$.
        \item $E$ is symmetric:
          Suppose $\models_\frakA Exy [s]$.
          Take the following sentence, which is valid in $\frakA$:
          \[ \forall x \forall y (Exy \lto (Exx \lto Eyx)). \]
          Since we have that $\forall x Exx$,
          for any such $x, y$ pair we have that:
          \begin{itemize}
            \item $\sbar(Exy) = T$
            \item $\sbar(Exx) = T$
          \end{itemize}
          we can deduce $Eyx$ as follows:
          \begin{align}
            &Exy \quad \text{(from choice of $x$ and $y$)} ~\label{1} \\
            &\forall x \forall y (Exy \lto (Exx \lto Eyx)) ~\label{2} \\
            &(Exx \lto Eyx) \quad \text{(modus ponens on ~\ref{1} and ~\ref{2})} ~\label{3} \\
            &Exx ~\label{4} \quad \text{(from the first sentence in the model)}\\
            &Eyx  \quad \text{(modus ponens on ~\ref{3} and ~\ref{4})} ~\label{5}
          \end{align}
        \item Transitive:
          the following sentence is valid in $\frakA$:
          \begin{align}
            \forall x \forall y \forall z (Exy \lto (Ez fx \lto Ez fy))~\label{eq:27.1}
          \end{align}
          \step
          Let $\models_\frakA Exy [s]$ and take $z \in \abs{\frakA}$
          such that $\models_\frakA Ezx$.
          Define a one-place function $f : \abs{\frakA} \to \abs{\frakA}$ by
          \[ f(a) = a \]
          Then, sentence ~\ref{eq:27.1} is equivalent to
          \[ \forall x \forall y \forall z (Exy \lto (Ezx \lto Ezy)). \]
          Since $E$ is symmetric, this means whenever $\models_\frakA Eyx$ and
          $\models_\frakA Exz$, then $\models_\frakA Eyz$, so $E$ is transitive.
      \end{enumroman}
    \item $P^\frakA$ is a well-defined relation on equivalence classes.
    
      \step
      Let $x, y \in \abs{\frakA}$ such that $\models_\frakA E xy[s]$.
      We show that $P^\frakA xa = P^\frakA ya$ for all $a \in \abs{\frakA}$.

      \step
      Suppose not, then there exists some $z \in \abs{\frakA}$ such that
      $\models_\frakA P xa[s]$ and $\not \models_\frakA P ya[s]$.

      \step
      Define a function $f : \abs{\frakA} \to \abs{\frakA}$ by
      \[ f(n) = E a n, \]
      then clearly $f(x) = T$ and $f(y) = F$, so $f(x) \ne f(y)$.

      \step
      However, we have the sentence
      \[ \forall x \forall y \forall z (Exy \lto (Ez fx \lto Ez fy)), \]
      which does not hold when
    \item $f^\frakA$ is a well-defined function on equivalence classes.
      
      \step
      Let $x, y \in \abs{\frakA}$ be such that $\models_\frakA Exy$.
      First, $E^\frakA$ is symmetric, so $E^\frakA xy \iff E^\frakA yx$.

      \step
      We also have the sentence
      $\forall x \forall y \forall z (Exy \lto (Ez fx \lto Ez fy))$
      as a valid axiom.
      But since $E^\frakA xy \iff E^\frakA yx$,
      whenever the above formula is valid the same for $Eyx$ is also valid,
      i.e. $\forall x \forall y \forall z (Eyx \lto (Ez fy \lto Ez fx))$.

      \step
      This means $\sbar(E z fx) = \sbar(E z fy)$ whenever $\sbar(E x y) = T$.
      Since $E^\frakA$ is transitive, this means $E (fx) (fy) = T$,
      therefore $f(x)$ and $f(y)$ are in the same equivalence class
      whenever $x$ and $y$ are in the same equivalence class
      so $f^\frakA$ is well-defined.
  \end{enumalph}
\end{Answer}

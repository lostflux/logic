\begin{problem}
  Let $\calL$ be the language for first-order logic with two-place
  predicate symbols $E$ and $P$ and one-place function symbol $f$.
  (We are not assuming that $\calL$ has the equality symbol.
  On the other hand, we are not ruling out the possibility that
  $\calL$ has the equality symbol and/or any number of parameter symbols
  in addition to $\forall$, $E$, $P$, and $f$.
  Other symbols are not relevant to this question.)

  \step
  Suppose $\frakA$ is a structure for $\calL$ that is a model of the sentence
  \[ \forall xExx \]
  and of every sentence of the form
  \[
    \forall x \forall y \forall z_1 \forall z_2 \ldots \forall z_n
    (Exy \lto (\alpha \lto \alpha')).
  \]
  where $\alpha$ is an atomic formula with variables included among
  $\set{x,y,z_1,z_2,\ldots,z_n}$, and $\alpha'$ is
  obtained from $\alpha$ by replacing some
  (possibly none, possibly some but not all, possibly all)
  occurrences of $x$ by $y$.
  Examples of sentences of this form are
  \[ \forall x \forall y (Exy \lto (Exx \lto Eyx))
  \text{ and }
  \forall x \forall y \forall z (Exy \lto (Ez fx \lto Ez fy)).
  \]
  An example of a sentence \emph{not} of this form is
  \[
    \forall x \forall y \forall z (Exy \lto (Ez fy \lto Ez fx)).
  \]

  \step
  Show that $E^\frakA$ is an equivalence relation on the universe $\abs{\frakA}$,
  that $P^\frakA$ induces a well-defined relation on equivalence classes, and that $f^\frakA$
  induces a well-defined function on equivalence classes.
\end{problem}
\begin{Answer}
  Let $s$ be a variable assignment satisfying the conditions as stated.
  Then:
  \step
  \begin{enumalph}
    \item $E^\frakA$ is an equivalence relation on the universe $\abs{\frakA}$.
      \begin{enumroman}
        \item $E$ is reflexive: This follows from the sentence $\forall x Exx$.
        \item $E$ is symmetric:
          Suppose $\models_\frakA Exy [s]$.
          Take the following sentence, which is valid in $\frakA$:
          \[ \forall x \forall y (Exy \lto (Exx \lto Eyx)). \]
          % Since we have that $\forall x Exx$,
          % for any such $x, y$ pair we have that
          % $\sbar(Exx) = T$. Therefore, the sentence reads
          % \[ \forall x \forall y (Exy \lto (T \lto Eyx)). \]

          % However, the sentence $(T \lto Eyx)$ is logically equivalent to
          % $(Eyx)$, since$(T \lto Eyx) = T$ \emph{iff} $Eyx = T$.

          % \step
          % Consequently, we have that
          % \[ \forall x \forall y (E xy \lto Eyx) \]
          % \begin{itemize}
          %   \item $\sbar(Exy) = T$
          %   \item $\sbar(Exx) = T$
          % \end{itemize}
          We can deduce $Eyx$ as follows:
          \begin{align}
            &Exy \quad \text{(from choice of $x$ and $y$)} ~\label{1} \\
            &\forall x \forall y (Exy \lto (Exx \lto Eyx)) \quad \text{(As derived above)} ~\label{2} \\
            &Exy \lto (Exx \lto Eyx) \quad \text{(Generalization theorem on ~\ref{2})} ~\label{3} \\
            &(Exx \lto Eyx) \quad \text{(modus ponens on ~\ref{1} and ~\ref{3})} ~\label{4} \\
            &\forall x Exx ~\label{5} \quad \text{(first sentence in the model)}\\
            &Exx \quad \text{(Generalization theorem on ~\ref{5})} ~\label{6} \\
            &Eyx  \quad \text{(modus ponens on ~\ref{4} and ~\ref{6})} ~\label{7}
          \end{align}
        \newpage
        \item Transitive:
          the following sentence is valid in $\frakA$:
          \begin{align}
            \forall x \forall y \forall z (Exy \lto (Ezx \lto Ezy))~\label{eq:27.1}
          \end{align}
          Since $E$ is symmetric;
          the formula is logically equivalent to
          \[ \forall x \forall y \forall z (\crim{Eyx} \lto (\crim{Exz} \lto \crim{Eyz})), \]
          so $E^\frakA$ is transitive.
      \end{enumroman}
    \item $P^\frakA$ is a well-defined relation on equivalence classes.
    
      \step
      Let $x, y \in \abs{\frakA}$ such that $\models_\frakA E xy[s]$.
      We show that $E^\frakA (P xa) (P ya)$ for all $a \in \abs{\frakA}$.

      \step
      Suppose not, then there exists some $z \in \abs{\frakA}$ such that
      $\lnot E^\frakA (P xa) (P ya)$, meaning $P xa$ and $P ya$ are not
      in the same equivalence class.
      However, we have a rule in our structure that
      \[ \forall x \forall y \forall z_1 \forall z_2 \ldots \forall z_n
      (Exy \lto (\alpha \lto \alpha')) \] where $\alpha'$ is obtained by
      replacing none, some, or all occurrences of $x$ in $\alpha$ with $y$.
      This means
      \[ \forall x \forall y \forall z_1 \forall z_2 \ldots \forall z_n
      (Exy \lto (P xa \lto P ya)) \] is a valid axiom in $\frakA$.
      Since $E$ is symmetric, $Exy$ implies $Eyx$, so
      we can also deduce that
      \[ 
        \forall x \forall y \forall z_1 \forall z_2 \ldots \forall z_n
        (Eyx \lto (P ya \lto P xa)).
      \]

      \step
      The two conditions only hold when $P ya$ and $P xa$
      are in the same equivalence class.

    \item $f^\frakA$ is a well-defined function on equivalence classes.
      
      \step
      Let $x, y \in \abs{\frakA}$ be such that $\models_\frakA Exy$.
      % First, $E^\frakA$ is symmetric, so $E^\frakA xy \iff E^\frakA yx$.

      \step
      The sentence
      \[ \forall x \forall y \forall z (Exy \lto (Ez fx \lto Ez fy)) \]
      as a valid axiom.

      \step
      Since $E$ is symmetric, $Eyx$ is true whenever $Exy$ is true,
      so the sentence
      \[ \forall x \forall y \forall z (Eyx \lto (Ez fy \lto Ez fx)) \]
      is also valid in $\frakA$.

      \step
      This means that whenever $E x y$ is satisfied,
      $E z fx$ is satisfied if and only if $E z fy$ is satisfied,
      so $fx$ and $fy$ are in the same equivalence class
      and $f^\frakA$ is well-defined on equivalence classes.
      
  \end{enumalph}
\end{Answer}

\begin{problem}
  Let $\Sigma$ be an effectively enumerable set of wffs.
  Assume that for each wff $\tau$, either $\Sigma \models \tau$
  or $\Sigma \models \lnot \tau$.
  Show that the set of tautological consequences of $\Sigma$ is decidable.
  \begin{enumalph}
    \item Do this where ``or'' is interpreted in the exclusive sense:
      either $\Sigma \models \tau$ or $\Sigma \models \lnot \tau$,
      but not both.
      \begin{Answer}
        The compactness theorem tells us that if $\Sigma \models \tau$,
        then there exists a finite subset $\Sigma_0$ such that $\Sigma_0 \models \set{\tau}$.
        
        \step
        Since $\Sigma$ is effectively enumerable,
        we can create an algorithm to list members of $\Sigma$,
        where $\sigma_k$ is the $k$th member of $\Sigma$ to be listed.
        We can then generate finite subsets of $\Sigma$ by taking an increasing set of the listed elements,
        say \[
          \Sigma_i = \begin{cases}
            \varnothing& \text{ if $i = 0$. }\\
            \Sigma_{i-1} \cup \set{\sigma_i} &\text{ otherwise. }
          \end{cases}
        \]

        \step
        For any arbitrary wff $\tau$, since either $\Sigma \models \tau$ or $\Sigma \models \lnot \tau$,
        there \emph{must} be some finite $\Sigma_k \subseteq \Sigma$ such that either $\Sigma_k \models \set{\tau}$
        or $\Sigma_k \models \set{\lnot \tau}$.
        When we find such a subset, we can mark $\tau$ as a tautological consequence of $\Sigma$
        or mark $\lnot \tau$ as a tautological consequence of $\Sigma$.
        If the condition does not yet hold, we can keep growing our subset.
      \end{Answer}
    \item Do this where ``or'' is interpreted in the inclusive sense:
      either $\Sigma \models \tau$ or $\Sigma \models \lnot \tau$,
      or both.
      \begin{Answer}
        There are two scenarios: the exclusive case and the inclusive case.
        In the exclusive case, we can proceed as in part $a$.
        In the inclusive case, the compactness theorem tells us that:
        \begin{enumroman}
          \item There must exist a finite $\Sigma_1 \subseteq \Sigma$ such that $\Sigma_1 \models \tau$.
          \item There must also exist a finite $\Sigma_2 \subseteq \Sigma$ such that $\Sigma_2 \models \lnot \tau$.
        \end{enumroman}

        \noindent
        Take $\Sigma_3 = \Sigma_1 \cup \Sigma_2 \subseteq \Sigma$.
        Then $\Sigma_3 \models \set{\tau}$ and $\Sigma_3 \models \set{\lnot \tau}$.

        \noindent
        Since no truth assignment may assign both $\tau$ and $\lnot \tau$,
        $\Sigma$ is not satisfiable and we may infer anything from the set $\Sigma$.
      \end{Answer}
  \end{enumalph}
\end{problem}

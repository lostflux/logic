\begin{problem}
  Show that from the corollary to the compactness theorem
  we can prove the compactness theorem itself
  (far more easily than we can starting from scratch).

  \begin{blockcolor}
    \textbf{Compactness Theorem:} A set of wffs is satisfiable
    iff every finite subset is satisfiable.

    \textbf{Corollary:} If $\Sigma \models \tau$, then there exists
    a finite subset $\Sigma_0$ such that $\Sigma_0 \models \set{\tau}$
  \end{blockcolor}
\end{problem}
\begin{Answer}
  % Suppose every finite subset $\Sigma_0 \subseteq \Sigma$ is satisfiable.

  \step
  For each element $\alpha \in \Sigma$, fix a finite set
  $\Sigma_\alpha \subseteq \Sigma$ such that $\alpha \in \Sigma_\alpha$.
  Since $\alpha \in \Sigma_\alpha$, $\Sigma_\alpha \cup \set{\lnot \alpha}$
  is not satisfiable, so $\Sigma_\alpha \models \alpha$.
  Therefore, any truth assignment $v$ that satisfies $\Sigma_\alpha$
  must assign $v(\alpha) = T$.

  \step
  If every such subset $\Sigma_\alpha$ is satisfiable
  for all elements $\alpha \in \Sigma$, then there exists a truth assignment
  $v$ such that $v(\alpha) = T$ for all $\alpha \in \Sigma$,
  implying that $\Sigma$ is satisfiable.

\end{Answer}


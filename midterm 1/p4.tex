\begin{problem}
  Define a relation $\equiv$ on $(\N^+)^2 = \set{(x, y) : x, y \in \N^+}$ by
  \[ (x_1, y_1) \equiv (x_2, y_2) \Iff x_1y_2 = x_2y_1. \]
  \begin{enumalph}
    \item Show that $\equiv$ is an equivalence relation on $(\N^+)^2$.
      \begin{Answer}
        To prove that $\equiv$ is an equivalence relation on $(\N^+)^2$,
        we need to show that it is reflexive, symmetric, and transitive.
        \begin{enumroman}
          \item \textbf{Reflexivity:}
            By definition, $(x_1, y_1) \equiv (x_2, y_2)$ if and only if $x_1 y_2 = x_2y_1$.
            For any arbitrary element $\alpha = (x, y)$, we always have that $x y = x y$
            so $\alpha \equiv \alpha$.
          \item \textbf{Symmetry:}
            Let $\alpha = (x_1, y_1)$ and $\beta = (x_2, y_2)$
            such that $\alpha \equiv \beta$, then $x_1 y_2 = x_2 y_1$.
            Since moving the left-hand side of the equation to the right-hand side
            and the right-hand side to the left-hand side does not change the equality,
            we also have that $x_2 y_1 = x_1 y_2$ so $\beta \equiv \alpha$.
          \item \textbf{Transitivity:}
            Let $a = (x_1, y_1)$, $b = (x_2, y_2)$, and $c = (x_3, y_3)$
            such that $a \equiv b$ and $b \equiv c$.
            Then $x_1 y_2 = x_2 y_1$ and $x_2 y_3 = x_3 y_2$.
            Therefore;
            \begin{align*}
              a \equiv b \Iff x_1 y_2 = x_2 y_1, \quad \text{so } \quad &\frac{x_1 y_2}{x_2 y_1} = 1 \\
              b \equiv c \Iff x_2 y_3 = x_3 y_2, \quad \text{ so } \quad &\frac{x_2 y_3}{x_3 y_2} = 1 \\
              &\Therefore \frac{x_1 y_2}{x_2 y_1} \cdot \frac{x_2 y_3}{x_3 y_2} = 1 \\
              &\Therefore \frac{x_1 \crim{y_2} \zaff{x_2} y_3}{\zaff{x_2} y_1 x_3 \crim{y_2}} = 1\\
              &\Therefore \frac{x_1 y_3}{x_3 y_1} = 1 \\
              &\Therefore x_1 y_3 = x_3 y_1 \\
              &\Therefore a \equiv c
            \end{align*}
        \end{enumroman}
      \end{Answer}
    \newpage
    \item Describe the equivalence class of $(3, 3)$\\
      \emph{Do this without mentioning the equivalence relation `$\equiv$'.}
      \begin{Answer}
        The equivalence class of $(3, 3)$ is the set of all pairs $(x, y)$
        such that $3x = 3y$.
        This is the set
        \[ \set{(x, y) : x, y \in \N^+ \text{ and } x = y}\]
      \end{Answer} 
    \item Suppose that we try to define a function on equivalence classes by
      \[ f[(x, y)] = [(2x^2, 2y^2)]. \]
      Either show that this function is well-defined or show that it is not.
      \begin{Answer}
        \begin{claim}
          This function is well-defined.

          \begin{proof}
            To show that $f$ is well-defined on the equivalence classes,
            we are going to show that any two members of an equivalence class
            are mapped to elements of the same equivalence class
            (although it may be different from the original equivalence class) ---
            i.e.., that $f(a) \equiv f(b)$ whenever $a \equiv b$.
            Since any two equivalence classes are always either equal or disjoint,
            this implies that $f[a] = [f(a)] = [f(b)] = f[b]$.

            \step
            Let $a = (x_1, y_1)$ and $b = (x_2, y_2)$ be two elements chosen from the same
            equivalence class without loss of generality, then $x_1 y_2 = x_2 y_1$.
    
            \step
            Let $[a'] = f[a] = [(2x_1^2, 2y_1^2)]$ and $[b'] = f[b] = [(2x_2^2, 2y_2^2)]$,
            such that $a' = (2x_1^2, 2y_1^2)$ and $b' = (2x_2^2, 2y_2^2)$.

            \step
            To show that $[a'] = [b']$, we need to show that $a' \equiv b'$ as follows:
            \begin{align*}
              2x_1^2 \cdot 2y_2^2 &= 4x_1^2 y_2^2 \\
              &= (2x_1y_2)^2 \\
              &= (2x_2y_1)^2 \quad \quad \text{ since $x_1y_2 = x_2 y_1$}\\
              &= 4x_2^2 y_1^2 \\
              &= 2x_2^2 \cdot 2y_1^2
            \end{align*}
    
            \step
            Therefore, whenever $a$ and $b$ are in the same equivalence class,
            $f[a] = f[b]$, so $f$ is well-defined.
            
            
          \end{proof}
        \end{claim}
      \end{Answer}
  \end{enumalph}
\end{problem}

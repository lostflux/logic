\section{Preliminary Questions}~\label{sec:prelims}

For this problem, we will start with the following definitions.
An element $g$ of a group $G$ has order $n$ if $n$ is the smallest
positive natural number such that
\[ n g = \underbrace{g + \ldots + g}_{\text{$n$ times}} = 0. \]
For example, $g$ has order $2$ if $g \neq 0$ and $g + g = 0$. \\
An element is said to be \emph{torsion free} if it does not have order $n$
for any $n \in \N$ with $n > 0$. A group is said to be \emph{torsion free}
if each of its elements, other than the identity, is torsion free. \\
Lastly, we say that a group $G$ is divisible if for each $g \in G$
and $n \in \N$ with $n > 0$, there exists $h \in G$ such that
\[ nh = \underbrace{h + \ldots + h}_{\text{$n$ times}} = g. \]

\subsection{Definitions}~\label{sec:definitions}

For convenience, we define the following shorthands used in the rest of the writing:
\begin{enumroman}
  \item $\Np = \N \setminus \{0\}$.
  \item $(x \neq y)$ is shorthand for $\lnot (x = y)$.
\end{enumroman}

\subsection{Axiomatizing Abelian Groups}~\label{sec:axiomatizing-abelian-groups}

\zaff{
  Define a language $\cL$ and a set of axioms $\Sigma$ such that any model
  that satisfies $\Sigma$ is an abelian group.\\ Next, define a set of axioms
  $T$ such that any model which satisfies $\Sigma \cup T$ is a
  divisible torsion free abelian group.
}

Using additive notation for groups, we define $\cL$ to specify
group operation ($+$) and the group identity, $0$.

\step
We specify $\cL$ as follows:
\begin{align}
  \cL &= \vector{ 0, +, -, = }
\end{align}

For any structure to be abelian, it needs to satisfy the group axioms
(closure, associativity, identity, and inversion) and commutativity.
We $\Sigma$ to contain the following axioms:

For any structure to be abelian, it needs to satisfy the following
group axioms:
\begin{enumroman}
  \item Closure. This does not need to be specified explicitly,
    since $n$-ary functions in first-order logic are always
    translated as $n$-ary functions from $\abs{\fA}^n$
    to a subset $S \subseteq \abs{\fA}$.
  \item Associativity:
    \begin{align}
      \forall g\, \forall h\, \forall i\, ( (g + (h + i)) = ((g + h) + i) ).
    \end{align}
  \item Existence of an identity element:
    \begin{align} \exists \eps\, \forall g ( g + \eps = g). \end{align}
    Using additive notation, we denote the identity element as $0$.
  \item Existence of inverses:
    \begin{align} \forall g\, \exists h\, (g + h = 0). \end{align}
    Using additive notation, we denote the inverse of $g$ as $-g$.

  \item For any structure to be abelian, it must be commutative.
    For this, we need an extra axiom:
    \begin{align} \forall g\, \forall h\, (g + h = h + g). \end{align}
\end{enumroman}

\subsection{Scalar Multiplication}~\label{sec:scalar-multiplication}

Let $n \in \Q$ be any rational number.
For any group element $h$, we define the multiplication of $h$ by $n$
as follows.
\begin{enumarabic}
  \item~\label{def:mult1} When $n$ is a natural number:
    \begin{enumroman}
      \item If $n > 0$, then $nh$ is the unique element
        \[ g = \underbrace{h + \ldots + h}_{\text{$n$ times}}. \]
      \item if $n = 0$, then $nh = 0g = 0$.
      \item If $n < 0$, then $nh$ is the unique element
        \[ g = \underbrace{(-h) + \ldots + (-h)}_{\text{$\abs{q}$ times}}. \]
    \end{enumroman}
  \item ~\label{def:mult2} When $n$ is not a natural number,
    write $n = \frac{a}{b}$, $a \in \N$ and $b \in \Np$
    (Since $n$ is rational, we can always do this).
    Then;
      \[ nh = \frac{a}{b} h = \frac{1}{b} (ah). \]
    First, note the multiplication $ah$ yields a unique element in the group
    by the definition of multiplication in~\ref{def:mult1} above.
    Call this element $h'$.
    We then define the multiplication of $h'$ by $\frac{1}{n}$ to be the
    element $g$ such that $ng = h'$.
    
    \begin{claim}
      If $nh_1 = nh_2 = g$, then $h_1 = h_2$.

      \begin{proof}
        Take $n, h_1, h_2$, and $g$ as in the claim.
        Then; 
        \begin{align*}
          nh_1 = nh_2 &= g \\
          \underbrace{h_1 + \ldots + h_1}_{\text{$n$ times}}
            = \underbrace{h_2 + \ldots + h_2}_{\text{$n$ times}} &= g \\
          \Therefore \parens{\underbrace{h_1 + \ldots + h_1}_{\text{$n$ times}}}
            - \parens{\underbrace{h_2 + \ldots + h_2}_{\text{$n$ times}}} = g - g &= 0 \\
          \Therefore \underbrace{(h_1 - h_2) + \ldots + (h_1 - h_2)}_{\text{$n$ times}} &= 0
        \end{align*}

        Since $G$ is torsion-free, we know that $ng \neq 0$ for any scalar
        $n \neq 0$ given $g \neq 0$.
        Therefore, it must be that $h_1 - h_2 = 0$.
        From the definition and axiomatizing of the group
        (section \ref{sec:axiomatizing-abelian-groups}),
        we know that we can deduce that $-h_2$ is equivalent to the unique element
        $-h_1$ such that $h_1 + (-h_1) = 0$,
        but if the inverse of $h_1$ and the inverse of $h_2$ are equivalent
        then $h_1$ and $h_2$ must be equivalent.

      \end{proof}
    \end{claim}

\end{enumarabic}

\newpage
\begin{claim}~\label{claim:scalar-multiplication}
  Scalar multiplication is well-defined.

  \begin{proof}
    
    Let $f : G \times \Q \to G$ be the function
    \[ f(h, n) = nh. \]
    That is, $f$ takes a group element $g \in G$ and a rational number $n \in \Q$
    and yields the element equivalent to the scalar multiplication
    of $g$ by $n$.\\
    $f$ is clearly well-defined for $n \in \N \subset \Q$,
    since the element $g = \underbrace{h + \ldots + h}_{\text{$n$ times}}$
    is unique for any $n \in \N$.

    Suppose $n$ is not a natural number.
    Since $n$ is rational, write $n = \frac{a}{b}$ and apply multiplication
    as defined in 
    \begin{enumroman}
    \item Suppose $h \in G$ and $m, n \in \N$ such that $f(h, m) = g$ and $f(h, n) = g$.
    Then
      \[
        \underbrace{h + \ldots + h}_{\text{$m$ times}}
        = \underbrace{h + \ldots + h}_{\text{$n$ times}} = g.
      \]
    Let $k = \max(m, n)$ and $l = \min(m, n)$, then $k - l \geq 0$, and:
    \begin{align*}
      kh - lh &= g - g = 0 \\
      \Therefore (k - l)\, h &= 0 \\
      \Therefore k - l &= 0 &\text{(since $G$ is torsion-free)} \\
      \Therefore n &= m
    \end{align*}

    \item Suppose $h_1, h_2 \in G$  and $n \in \Np$ such that
    $f(h_1, n) = g$ and $f(h_2, n) = g$.
    Then
    \begin{align*}
        \underbrace{h_1 + \ldots + h_1}_{\text{$n$ times}}
        &= \underbrace{h_2 + \ldots + h_2}_{\text{$n$ times}} = g \\
      \Therefore \parens{\underbrace{h_1 + \ldots + h_1}_{\text{$n$ times}}}
      - \parens{\underbrace{h_2 + \ldots + h_2}_{\text{$n$ times}}} &= g - g = 0 \\
      \Therefore \underbrace{(h_1 - h_2) + \ldots + (h_1 - h_2)}_{\text{$n$ times}} &= 0
    \end{align*}
    Since $G$ is torsion-free, $ng \neq 0$ for any $g \in G$ given $g \neq 0$.
    Therefore, if $n(h_1 - h_2) = 0$, it follows that $h_1 -h_2 = 0$,
    so $h_1 = h_2$.
  \end{enumroman}
  \step
  Therefore, scalar multiplication is well-defined.
  \end{proof}

\end{claim}
\subsection{Divisibility and Torsion-Free~ness}~\label{sec:div-torsion-free}

We define an extra set of conditions, $T$,
which must be satisfied (in addition to those in $\Sigma$)
for a structure to be a divisible torsion-free abelian group:

\begin{align}
  T_1 &= \set{ \forall g\, (g \neq 0 \lto ng \neq 0) \mid n = 1, 2, 3, \ldots } \\
  T_2 &= \set{ \forall g\, \exists h,\, (g = nh) \mid n = 1, 2, 3, \ldots } \\
  T &= T_1 \cup T_2
\end{align}

\bigskip

\subsection{Existence of \texorpdfstring{$\Q$}{Q}-vector space structures}~\label{sec:q-vector-space}

\zaff{
  Show that any divisible torsion free abelian group
  has a $\Q$-vector space structure. \\
  \emph{Hint:} Show that if G is such a group, $n \in \N$ with $n > 0$
  and $g \in G$ then there is a unique $h \in G$ such that $nh = g$.
  Note that to show there is a $\Q$-vector space structure,
  you must define scalar multiplication (and prove it is well-defined).
}

\subsubsection{Scalar Multiplication (Definition)}~\label{subsec:scalar-multiplication}


\subsection{Any divisible torsion-free abelian group has a
\texorpdfstring{$\Q$}{Q}-vector space structure}~\label{sec:torsion-free-q-vector-space}

As seen in the proof to \ref{claim:scalar-multiplication}, if a group $G$
is divisible and torsion-free, then scalar multiplication is well-defined.
Since $G$ is divisible, then for every $g \in G$ and $n \in \Np$
there is a unique $h \in G$ such that $nh = g$.
As seen in the proof to \ref{claim:scalar-multiplication},
for fixed $n$, any such $h$ must be unique
(likewise, for fixed $h$, any such $n$ must be unique).
Therefore, $G$ has a $\Q$-vector space structure.


\bigskip

\subsection{Axiomatizing Abelian Groups Wherein Each Element Other Than the Identity
Has Order 2}~\label{sec:axiomatizing-abelian-groups-order-two}

\zaff{
  Define a set of axioms $S$ such that any model that satisfies $\Sigma \cup S$
  is an abelian group in which each element other than the identity has order two.
  Can we give a model for $\Sigma \cup S$ a vector space structure? \\
  \emph{Hint:} Be creative in your choice of the scalar field.
}

We define $S$ to contain the following axioms:
\begin{align}
  \forall g\, ( (g \neq 0) \lto ((g + g) &= 0) ) &\text{(every element has order 2)}
\end{align}

\bigskip
For examples of models of $\Sigma \cup S$;
first, let's consider finite abelian groups in which every non-identity
element has order 2.
The smallest such group is the cyclic group of order $2$:
\begin{align*}
  C_2 &= \parens{ \set{0, 1}, + }
\end{align*}

To generate larger groups where every non-identity element has order 2,
we can take the direct product of $C_2$ with itself.
For instance, the Klein-Four group $V_4$ is isomorphic to $C_2 \times C_2$.
For an infinite such group, take an infinite sequence of direct products of $C_2$:
\begin{align*}
  C_2^{\infty} &= C_2 \times C_2 \times C_2 \times \ldots
\end{align*}
Take any element $g \in C_2^{\infty}$ such that $g$ is not the identity.
Since $g$ has order $2$, we have that $3g = 2g + g = 0 + g = g$,
and, more generally, for all $n \in \Np$, $ng = k g$ where $n \equiv k \mod 2$.
Therefore, $G$ cannot have a $\Q$-vector space structure under scalar multiplication
with elements of $\Np$.
One work-around is to limit the scalar field to a two-element field,
such as $\F_2$.
\[ \F_2 = \parens{\set{0, 1}, +, \times}. \]
Under multiplication with elements of $\F_2$, we have that
$\forall g (0g = 0)$ and $\forall g (1g = g)$, thus we have a
$\Q$-vector space structure.

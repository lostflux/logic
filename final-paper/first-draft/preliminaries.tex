\section{Introduction}~\label{sec:prelims}
Mathematical logic is interesting in itself, but
its ability to effectively model and describe other
fields in mathematics is, perhaps, even more intriguing
and what makes the study of logic so useful to the regular
mathematician.
For instance, using logic one may model aspects of
analysis, algebra, calculus, probability, geometry,
and even game theory.
This offers disparate and often insightful perspectives into these other fields,
letting one apply the laws of logic to their study of the structures
of other fields in mathematics and the resulting consequences.
More importantly, modeling other structures is useful in the study of logic
itself as it lets one interrogate the rules of logic from other perspectives.

In this paper, we will model the structure of abelian groups using
first-order logic. We will then narrow down our study to groups that
are \emph{abelian}, \emph{divisible}, and \emph{torsion-free}.
We will then study groups in this category as vector-spaces
over the rational numbers, $\Q$, and study how the structure of these
vector spaces can tell us about logic.

\section{Definitions}~\label{sec:definitions}
\emph{
  Note: This section is included mostly as a prelude for the
  non-experienced mathematician, and may be skipped by readers
  well-versed in
  \textbf{Group Theory},
  \textbf{Abstract Vector Spaces},
  and \textbf{First-Order Logic}.
}

\subsection{Groups}~\label{sec:def-groups}

A group $G = (S, \ast)$ is a set $S$ paired with a binary operation,
$\ast : S \to S$, satisfying the following axioms:
\begin{enumroman}
  \item Closure: $x \ast y \in S$ for any two elements $x, y \in S$.
  \item Associativity: $(x \ast y) \ast z = x \ast (y \ast z)$
    for any three elements $x, y, z \in S$.
  \item Identity: there exists an element $\eps \in S$ such that
    for any $x \in S$, $x \ast e = e \ast x = x$.
  \item Inversion: for any $x \in S$, there exists an element
    $x^{-1} \in S$ such that $x \ast x^{-1} = x^{-1} \ast x = \eps$.
\end{enumroman}
There are two common notations for groups:
\begin{enumroman}
  \item Additive notation: $G = (S, +, 0)$, where $0$ is the identity
    element and $+$ is the binary operation.
    For any element $g$, we denote the inverse of $g$ as $-g$.
  \item Multiplicative notation: $G = (S, \cdot, 1)$, where $1$ is the
    identity element and $\cdot$ is the binary operation.
    For any element $g$, we denote the inverse of $g$ as $g^{-1}$.
\end{enumroman}

Indeed, some groups with numerical elements use modified versions of
addition and multiplication as their binary operations.
For instance, the group of integers modulo a positive integer $n$
is denoted as $\Z/n\Z$, where $a + b$ in the group is equivalent
to $a + b \mod n$.
Multiplicative groups can also be formed this way, but require
more attention since not every element less than some integer $n$
necessarily has an inverse. We denote such groups as $(\Z/n\Z)^\times$
and usually include only the elements co-prime to $n$ ---
for instance, $(\Z/12\Z)^\times$ contains the elements $1, 5, 7$, and $11$,
with each element being its own inverse.

In this paper, we will use additive notation for groups.

\subsection{Fields}~\label{sec:def-fields}

A field $F = (S; 0, 1, +, \times)$ is a set $S$ with two binary operations, $+$ and $\cdot$,
such that the pairing $(S, +)$ is an \emph{abelian} group
and the pairing $(S \setminus \set{ 0 }, \cdot)$ is a 
group (not necessarily abelian). Some of the most common examples of a field
is the real numbers $R = (\R, 0, 1, +, \cdot)$, the rational numbers $Q = (\Q, 0, 1, +, \cdot)$,
and the complex numbers $C = (\C, (0, 0), (1, 0), +, \cdot)$ under their standard addition and multiplication
operations.


\subsection{Vector Spaces over Fields}~\label{sec:def-vector-spaces}

A vector space $V = (S, +)$ over a field $F$ is an \emph{abelian group}
such that multiplication of elements of $V$ by elements of the field $F$
is defined and satisfies the following axioms:
\begin{enumroman}
  \item $0 x = 0$ for any $x \in S$ and the additive identity $0 \in F$.
  \item $1 x = x$ for any $x \in S$ and the multiplicative identity $1 \in F$.
  \item $n x = \underbrace{x + \ldots + x}_{\text{$n$ times}}$
    for any $x \in S$ and any $n \in F$.
  \item $(-n) x = \underbrace{-x + \ldots + -x}_{\text{$n$ times}}$
    for any $x \in S$ and any $n \in F$.
  \item $n^{-1} x = y$ where $y$ is the element in the field
    such that $n y = x$.
\end{enumroman}

\subsection{Divisible and Torsion Free Groups}~\label{sec:def-div-groups}

The order of any element $g$ in a group $G$ is the smallest positive
integer $n$ such that $n g = 0$.
For instance, if $g + g + g = 0$, then $g$ has order $3$.

\begin{enumarabic}
  \item A group $G$ is \emph{divisible} if for any $g \in G$
    and any $n \in \N$ with $n > 0$, there exists an element $h \in G$
    such that $n h = g$.
  \item An element $g \in G$ is \emph{torsion free} if $ng \neq 0$
    for any positive integer $n > 0$,
    and a group is \emph{torsion free} if every non-identity element is
    torsion free.
\end{enumarabic}



% For this problem, we will start with the following definitions.
% An element $g$ of a group $G$ has order $n$ if $n$ is the smallest
% positive natural number such that
% \[ n g = \underbrace{g + \ldots + g}_{\text{$n$ times}} = 0. \]
% For example, $g$ has order $2$ if $g \neq 0$ and $g + g = 0$. \\
% An element is said to be \emph{torsion free} if it does not have order $n$
% for any $n \in \N$ with $n > 0$. A group is said to be \emph{torsion free}
% if each of its elements, other than the identity, is torsion free. \\
% Lastly, we say that a group $G$ is divisible if for each $g \in G$
% and $n \in \N$ with $n > 0$, there exists $h \in G$ such that
% \[ nh = \underbrace{h + \ldots + h}_{\text{$n$ times}} = g. \]

% % \subsection{Definitions}~\label{sec:definitions}

% For convenience, we define the following shorthands used in the rest of the writing:
% \begin{enumroman}
%   \item $\Np = \N \setminus \{0\}$.
%   \item $(x \neq y)$ is shorthand for $\lnot (x = y)$.
% \end{enumroman}

\newpage
\section{Axiomatizing Groups}~\label{sec:axiomatizing-abelian-groups}
To model groups using logic, we need to specify a sufficiently capable
language. The language dictates which symbols are valid in sentences/formulas
we use to describe groups. We will use the following language:
\begin{align}
  \cL &= \vector{ 0, +, -, = }
\end{align}

\subsection{Abelian Groups}~\label{sec:axiom-abelian-groups}

For any structure to be an abelian group, it needs to satisfy the group axioms
(closure, associativity, identity, and inversion) and commutativity
(see section~\ref{sec:def-groups} for reference).
We define the set $\Sigma$ to contain the following abelian group axioms:
\begin{align}
  \forall g\, \forall h\, &(g + h = h + g) &\text{ (commutativity)} \\
  \forall g\, \forall h\, \forall i\, &( (g + (h + i)) = ((g + h) + i) )
    &\text{ (associativity)} \\
  \exists \eps\, \forall g &( g + \eps = g) &\text{ (group identity)} \\
  \forall g\, \exists h\, &(g + h = 0) &\text{ (group inversion)}
\end{align}
Since $n$-ary functions in a structure $\fA$ in first-order logic always assume
the domain $\abs{\fA}^n$, and the range $\abs{\fA}$, we do not need to
explicitly specify an axiom for closure.


\subsection{Divisible, Torsion Free Abelian Groups}~\label{sec:axioms-div}

We define an extra set of conditions
which must be satisfied (in addition to those in $\Sigma$)
for a structure to be a divisible torsion-free abelian group.
Recall that a group $G$ is \emph{divisible} if for any $g \in G$
and any $n \in \N$ with $n > 0$, there exists an element $h \in G$
such that $n h = g$.
We can represent this using an infinite set of axioms
\[ T_1 = \set{ \forall g\, \exists h,\, (g = nh) \mid n = 1, 2, 3, \ldots }. \]

Recall that a group $G$ is \emph{torsion free} if every non-identity element
is torsion free. That is, if $g \neq 0$ then $ng \neq 0$ for any positive
integer $n > 0$. We can represent this using an infinite set of axioms
\[ T_2 = \set{ \forall g\, (g \neq 0 \lto ng \neq 0) \mid n = 1, 2, 3, \ldots }. \]

Let $T = T_1 \cup T_2$, then any model satisfying $\Sigma \cup T$
is a divisible, torsion free abelian group.



\subsection{Abelian Groups Wherein Each Element Other Than the Identity
Has Order 2}~\label{sec:axiom-order-two}

To specify that each element in an abelian group has order $2$,
define the set
\[ S = \set{\forall g\, ( g \neq 0 \lto (g + g = 0) ) }, \]
then any model satisfying $\Sigma \cup S$ is an abelian group
wherein each element other than the identity has order $2$.


\section{Vector Space Structures}~\label{sec:vec-space-structures}
A group $G$ has a \emph{vector space structure} over some field $F$
if multiplication of elements of $G$ by scalars in $F$ is defined

\subsection{Divisible, Torsion Free Abelian Groups}
~\label{sec:adiv-torsion-free-abelian-groups}

\begin{claim}~\label{claim:mult}
  The scalar multiplication of elements of a divisible, torsion free
  abelian group $G$ by elements of the field $\Q$ is well-defined.
  
  \begin{proof}
    The proof for integer multiplication is trivial since
    the sum $\underbrace{x + \ldots + x}_{\text{$n$ times}}$
    always names a unique element in the vector space
    (and in the equivalent abelian group).
    Consider multiplication by non-integer rational numbers.
    Take $n \in \Q$ and $h_1, h_2, g \in G$ such that $nh_1 = nh_2 = g$,
    then;
    \begin{align*}
      nh_1 = nh_2 &= g \\
      \underbrace{h_1 + \ldots + h_1}_{\text{$n$ times}}
        = \underbrace{h_2 + \ldots + h_2}_{\text{$n$ times}} &= g \\
      \Therefore \parens{\underbrace{h_1 + \ldots + h_1}_{\text{$n$ times}}}
        - \parens{\underbrace{h_2 + \ldots + h_2}_{\text{$n$ times}}} = g - g &= 0 \\
      \Therefore \underbrace{(h_1 - h_2) + \ldots + (h_1 - h_2)}_{\text{$n$ times}} &= 0
    \end{align*}
    But $G$ is torsion-free, meaning $ng \neq 0$ for any scalar $n \neq 0$
    given $g \neq 0$. so $h_1 - h_2$ must be equivalent to $0$.
    Group inverses are unique, so this means that $-h_1$ and $-h_2$ are
    the same element since $h_1 - h_2 = 0$.
    It then follows that $h_1$ and $h_2$ are equivalent.
  \end{proof}
\end{claim}

\begin{claim}~\label{claim:div-vec-space}
  Any divisible torsion-free abelian group has a $\Q$-vector space structure

  \begin{proof}
    Let $G$ be a divisible, torsion-free abelian group,
    and let $g \in G$.
    By definition, since $G$ is divisible, then for every $n \in \N$
    such that $n > 0$, there exists some element $h \in G$ such that
    $nh = g$. Furthermore, as seen in the proof to \ref{claim:mult},
    if $nh_1 = g$ and $nh_2 = g$ then $h_1 = h_2$
    --- hence, for any $g \in G, n \in \N, n > 0$, not only does the
    element $h$ exist such that $nh = g$, but the element $h$
    is also unique.
  \end{proof}
\end{claim}
Therefore, any divisible, torsion free abelian group
forms a vector space over multiplication by rational numbers
(i.e. has a $\Q$-vector space structure).

\begin{claim}
  $\Sigma \cup T$ is categorical over the cardinality of the reals.

  \begin{proof}
    Any model $\fA$ of $\Sigma \cup T$ is a divisible,
    torsion free abelian group.
    As shown in the proof to \ref{claim:div-vec-space},
    any such abelian group has a $\Q$-vector space structure.
    Define an isomorphism $\phi$ by: $\phi : \R \to \abs{\fA}$ by
    \begin{align*}
      \phi : \R &\to \abs{\fA} \\
      0 &\mapsto 0^\fA \\
      1 &\mapsto 1^\fA \\
      n &\mapsto n^\fA \text{ for } n \in \N \\
      \frac{p}{q} &\mapsto \frac{1}{q} \cdot p^\fA \text{ for } p, q \in \N
    \end{align*}
    $\phi$ must be a bijection since every element in $G$ has infinite order
    and every element in $G$ is divisible by every rational number in $\Q$,
    hence the cardinality of $\abs{\fA} = \abs{\R}$.

    Therefore, there is, up to isomorphism, a unique model such model
    $\fA$ of $\Sigma \cup T$ that has the size of $\abs{\R}$.
  \end{proof}
\end{claim}

\begin{claim}
  $\Sigma \cup T$ is not countably categorical.

  \begin{proof}
    Let $\fA$ be a model for $\Sigma \cup T$.
    Then $\fA$ is a divisible, $\abs{\fA}$ cannot have a countable
    cardinality. Supposing it was, then every element has a successor.
    Let $x$ be the successor of $0$, and take $n > 1$ then there exists
    an element $y$ such that $ny = x$, which implies that there exists
    an element between $0$ and $x$, contradicting the
    assumption that $x$ is the successor of $0$.
  \end{proof}
\end{claim}


\bigskip

\subsection{Abelian Groups Wherein Each Element Other Than the Identity
Has Order 2}~\label{sec:axiomatizing-abelian-groups-order-two}

We define $S$ to contain the following axioms:
\begin{align}
  \forall g\, ( (g \neq 0) \lto (g + g &= 0) ) &\text{(every element has order 2)}
\end{align}

\bigskip
For examples of models of $\Sigma \cup S$;
first, let's consider finite abelian groups in which every non-identity
element has order 2.
The smallest such group is the cyclic group of order $2$:
\begin{align*}
  C_2 &= \parens{ \set{0, 1}, 0, + }
\end{align*}

To generate larger groups where every non-identity element has order 2,
we can take the direct product of $C_2$ with itself.
For instance, the Klein-Four group $V_4$ is isomorphic to $C_2 \times C_2$.
For an infinite such group, take an infinite sequence of direct products of $C_2$:
\begin{align*}
  C_2^{\infty} &= C_2 \times C_2 \times C_2 \times \ldots
\end{align*}
Take any element $g \in C_2^{\infty}$ such that $g$ is not the identity.
Since $g$ has order $2$, we have that $3g = 2g + g = 0 + g = g$,
and, more generally, for all $n \in \Np$, $ng = k g$ where $n \equiv k \mod 2$.
Therefore, $G$ cannot have a $\Q$-vector space structure under scalar multiplication
with elements of $\Np$.
One work-around is to limit the scalar field to a two-element field,
such as $\F_2$.
\[ \F_2 = \parens{\set{0, 1}, +, \times}. \]
Under multiplication with elements of $\F_2$, we have that
$\forall g (0g = 0)$ and $\forall g (1g = g)$, thus we have a
$\Q$-vector space structure.

\section{Conclusions}~\label{sec:conclusions}

These are all the conclusions.



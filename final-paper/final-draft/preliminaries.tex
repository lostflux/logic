\section{Introduction}~\label{sec:prelims}
Mathematical logic is interesting in itself, but
its ability to effectively model and describe other
fields in mathematics is, perhaps, even more intriguing
and what makes the study of logic so useful to the regular
mathematician.
For instance, using logic one may model aspects of
analysis, algebra, calculus, probability, geometry,
and even game theory.
This offers disparate and often insightful perspectives into these other fields,
letting one apply the laws of logic to their study of the structures
of other fields in mathematics and the resulting consequences.
More importantly, modeling other structures is useful in the study of logic
itself as it lets one interrogate the rules of logic from other perspectives.

In this paper, we will model the structure of abelian groups using
first-order logic. We will then narrow down our study to groups that
are emph{abelian}, \emph{divisible}, and \emph{torsion-free}.
We will then study groups in this category as vector-spaces
over the rational numbers, $\Q$, and study what the structure of these
vector spaces can tell us about logic.











\newpage
\section{Preliminaries}~\label{sec:preliminaries}
\emph{
  Note: This section is included mostly as a prelude for the
  non-experienced mathematician, and may be skipped by readers
  well-versed in
  \textbf{Group Theory},
  \textbf{Abstract Vector Spaces},
  and \textbf{First-Order Logic}.
}

\subsection{Groups}~\label{sec:def-groups}

\begin{definition}
  A group $G = (S, \ast)$ is a set $S$ paired with a binary operation,
  $\ast : S \to S$, satisfying the following axioms:
  \begin{enumroman}
    \item Closure: $x \ast y \in S$ for any two elements $x, y \in S$.
    \item Associativity: $(x \ast y) \ast z = x \ast (y \ast z)$
      for any three elements $x, y, z \in S$.
    \item Identity: there exists an element $\eps \in S$ such that
      for any $x \in S$, $x \ast e = e \ast x = x$.
    \item Inversion: for any $x \in S$, there exists an element
      $x^{-1} \in S$ such that $x \ast x^{-1} = x^{-1} \ast x = \eps$.
  \end{enumroman}
  There are two common notations for groups:
  \begin{enumroman}
    \item Additive notation: $G = (S, 0, +)$, where $0$ is the identity
      element and $+$ is the binary operation.
      For any element $g$, we denote the inverse of $g$ as $-g$.
    \item Multiplicative notation: $G = (S, 1, \cdot)$, where $1$ is the
      identity element and $\cdot$ is the binary operation.
      For any element $g$, the inverse of $g$ is denoted as $g^{-1}$.
  \end{enumroman}

  However, the group operation need not be normal addition or multiplication.
  For instance, the group of integers modulo a positive integer $n$,
  denoted $\Z/n\Z$, defines its binary operation on any two elements $a$ and $b$
  as $(a + b) \mod n$ --- so, for instance, $5 + 7 = 0$ in $\Z/12\Z$.
  More absurd operations can be defined in some groups, including
  rotations and reflections.\\
  \emph{Henceforth, we will use additive notation for groups.}
\end{definition}

\begin{definition}
  A group $G$ is \emph{abelian} if $x + y = y + x$ for any two
  elements $x, y \in G$.
\end{definition}

\begin{definition}
  The order of any element $g$ in a group $G$ is the smallest positive
  integer $n$ such that $n g = 0$.
  For instance, if $g + g + g = 0$, then $g$ has order $3$.
\end{definition}


\subsection{Fields}~\label{sec:def-fields}

A field $F = (S; 0, 1, +, \times)$ is a set $S$ with two binary operations, $+$ and $\cdot$,
such that the pairing $(S, +)$ is an \emph{abelian} group
and the pairing $(S \setminus \set{ 0 }, \cdot)$ is also an abelian group.
Some common examples of fields are the real numbers $\fR = (\R, 0, 1, +, \cdot)$,
the rational numbers $\fQ = (\Q, 0, 1, +, \cdot)$,
and the complex numbers $\fC = (\C, (0, 0), (1, 0), +, \cdot)$.


\subsection{Vector Spaces over Fields}~\label{sec:def-vector-spaces}

A vector space $V$ over a field $F$ is an \emph{abelian group}
whose elements can be multiplied by elements in the field.
We refer to the elements of the vector space as \emph{vectors}
and the elements of the field as \emph{scalars}.
The multiplication of a vector by a scalar is called \emph{scaling}.
For any vector space $V$ over a field $F$, the scaling operation
has to satisfy the following axioms:
\begin{enumroman}
  \item $1 x = x$ for any $x \in V$, where $1$ is the multiplicative identity in $F$.
  \item $a(bx) = (ab)x$ for any vector $x \in V$ and any two scalars $a, b \in F$.
  \item $(a+b)x = ax + bx$ for any vector $x \in V$ and any two scalars $a, b \in F$.
  \item $a(x+y) = ax + ay$ for any two vectors $x, y \in V$ and any scalar $a \in F$.
  % \item 
  % \item $0 x = 0$ for any $x \in S$ and the additive identity $0 \in F$.
  % \item $n x = \underbrace{x + \ldots + x}_{\text{$n$ times}}$
  %   for any $x \in S$ and any $n \in F$.
  % \item $(-n) x = \underbrace{-x + \ldots + -x}_{\text{$n$ times}}$
  %   for any $x \in S$ and any $n \in F$.
  % \item $n^{-1} x = y$ where $y$ is the element in the field
  %   such that $n y = x$.
\end{enumroman}

% \newpage
\begin{claim}~\label{claim:scale-neg-1}
  Scaling any vector by $-1$ yields the additive inverse of that vector.
  Equivalently, $-1 x = -x$ for any $x \in V$.

  \begin{proof}
    By definition, $-1$ is the additive inverse of $1$ in $F$,
    where $1$ is the multiplicative identity.
    Thus, \[ -1 x = -(1 x) = -(x) = -x. \]
  \end{proof}

  \begin{corollary}
    Scaling any vector by $-n$ yields the same result as scaling the inverse
    of that vector by $n$.
    Equivalently, $(-n) x = n (-x)$ for any $x \in V$ and any $n \in F$.

    \begin{proof}
      \[ (-n) x = (n \cdot (-1)) x = n \cdot (-1) (x) = n (-x). \]
    \end{proof}
  \end{corollary}

  \begin{corollary}
    Scaling any vector $x \in V$ by $-n$ yields the additive inverse to the result
    of scaling the same vector by $n$.

    \begin{proof}
      By definition of scalar multiplication of vectors;
      \[
        (-n) x = (-1 \cdot n) x = (-1) (n x) = - (nx)
      \]
    \end{proof}
  \end{corollary}

  \begin{corollary}
  Scaling any vector by $0$ yields the zero vector.
  Equivalently, $0x = 0$ for any $x \in V$.

  \begin{proof}
    By writing $0x = (a + (-a))x$ for any element $a \in F$, we have:
    \begin{align*}
      0x &= (a + (-a))x
        = ax + (-ax)
        = 0 
    \end{align*}
  \end{proof}
  \end{corollary}
\end{claim}

\newpage
\subsection{Divisible and Torsion Free Groups}~\label{sec:def-div-groups}

\begin{definition}
  Where $F$ is a field with numeric elements and $V$ is a vector space over $F$,
  define scalar multiplication in accordance with the vector space axioms
  as follows:
  
  \begin{align}
    nx &= \begin{cases}~\label{def:scalar-multiplication}
      0 & \text{if $n = 0$.} \\
      \underbrace{x + \ldots + x}_{\text{$n$ times}} & \text{if $n$ is an integer and $n > 0$.} \\
      \underbrace{-x + \ldots + -x}_{\text{$n$ times}} & \text{if $n$ is an integer and $n < 0$.} \\
      y \quad \text{ where } x = qy \quad & \text{if $n = q^{-1}$ for some integers $q \in F$.} \\
      p \parens{q^{-1} \cdot x} &\text{if $n = pq^{-1}$ with $q \neq 0$ amd $q \nmid p$.}
    \end{cases}
  \end{align}
\end{definition}

A dangling question from the above definition is whether for any value of $n$
we can always express an element $g$ as $nh$ for some other element $h$.
\begin{definition}
  We say a group $G$ is \emph{divisible} if for any $g \in G$
  and any $n \in \N$ with $n > 0$, there exists an element $h \in G$
  such that $n h = g$.
\end{definition}
\begin{definition}
  An element $g \in G$ is \emph{torsion free} if $ng \neq 0$
    for any positive integer $n > 0$,
    and a group is \emph{torsion free} if every non-identity element is
    torsion free.
\end{definition}

\newpage
\section{Analysis of some Vector Space Structures}~\label{sec:analysis-vec-space-structures}
Recall that for any arbitrary structure $\fA$ to be a vector space over a field $F$,
$\fA$ must be an \emph{abelian} group, and the multiplication of elements of $\fA$
by elements of $F$ must be well-defined and satisfy the vector space axioms.
When both of these conditions are satisfied,
we say that $\fA$ has a vector space structure over $F$.
In this section, we shall see two examples of abelian groups that have a vector space
structure over some field and look at why they do not have a vector space structure
over some other fields.

\subsection{Divisible, Torsion Free Abelian Groups over \texorpdfstring{$\Q$}{Q}}
~\label{sec:div-torsion-free-abelian-groups}

Recall that an abelian group $G$ is divisible if for every $n \in \N_{>0}$,
every element $g \in G$ element can be written as $nh$ for some other element $h \in G$.
Furthermore, a group $G$ is torsion free if every non-identity element is torsion free
--- that is, if $g$ is not the identity element, then scaling it by any non-zero
factor $n \in \N_{>0}$ yields a non-zero element.

Note that in the field $\Q$, the additive inverse of any element $g$
is the element $-g$ and the multiplicative inverse of any element $g$ is the element $1/g$
 --- for example, the additive inverse inverse of $7$ is $-7$,
and the multiplicative inverse of $7$ is $1/7$.

\begin{claim}~\label{claim:torsion-free-mult}
  The scalar multiplication of elements of a divisible, torsion free
  abelian group $G$ by elements of the field $\Q$ is well-defined.
  \begin{proof}
    Using the definition of multiplication(~\ref{def:scalar-multiplication}):
    \begin{enumarabic}
      \item The proof for multiplication by positive integers is straightforward since
        the sum $\underbrace{x + \ldots + x}_{\text{$n$ times}}$
        always names a unique element in the group.
      \item The proof for multiplication by negative integers is also straightforward
        since the sum $\underbrace{-x + \ldots + -x}_{\text{$n$ times}}$
        always names a unique element in the group.
      \item The proof for multiplication by $0$ is trivial;
        $0 \cdot x$ always evaluates to $0$.
      \item For multiplication by a rational number $n = p/q$,
        we have $nh = (p/q)h = p(\frac{1}{q} \cdot h)$.
        $p$ is an integer, so the multiplication by $p$ is well-defined as above.
        For multiplication by $\frac{1}{q}$, we define $\frac{1}{q}\cdot g
        = h$ such that $g = qh$. For multiplication to be well-defined, we must show that
        any such element $h$ is unique, i.e. if $qh_1 = qh_2$ then we necessarily have that
        $h_1 = h_2$.
        \begin{align*}
          qh_1 = qh_2 &= g \\
          \underbrace{h_1 + \ldots + h_1}_{\text{$q$ times}}
            = \underbrace{h_2 + \ldots + h_2}_{\text{$q$ times}} &= g \\
          \Therefore \parens{\underbrace{h_1 + \ldots + h_1}_{\text{$q$ times}}}
            - \parens{\underbrace{h_2 + \ldots + h_2}_{\text{$q$ times}}} = g - g &= 0 \\
          \Therefore \underbrace{(h_1 - h_2) + \ldots + (h_1 - h_2)}_{\text{$q$ times}} &= 0 \\
          \Therefore h_1 - h_2 &= 0 &\text{ (since $G$ is torsion free)} \\
          \Therefore h_1 &= h_2
        \end{align*}
    \end{enumarabic}
  \end{proof}
\end{claim}

\begin{corollary}
  Any divisible, torsion free abelian group has a vector space structure over $\Q$.

  \begin{proof}
    This follows from the proof to claim ~\ref{claim:torsion-free-mult}
  \end{proof}
\end{corollary}










\subsection{Abelian Groups Wherein Each Non-Identity Element Has Order 2}
~\label{sec:abelian-groups-order-two}

\begin{claim}~\label{claim:abelian-order-two-finite-isomorphic}
  Any two abelian groups of the same finite order $n$ in which all non-identity elements
  have order $2$ are isomorphic.

  \begin{proof}
    Let $G$ and $H$ be two such groups.
    Since $G$ and $H$ have the same order $n$, they each have exactly $n-1$ elements
    of order $2$.
    To construct an isomorphism between $G$ and $H$, map the identity element in $G$
    to the identity element in $H$, and map each non-identity element $g \in G$
    to some nonidentity element $h \in H$ such that no two elements are mapped to the same element.
    This isomorphism is always constructible since there's exactly $n-1$ such elements
    in each of $G$ and $H$.

    If every element in a group has order $2$, and the group has finite order,
    then the group must have an order that is a power of $2$
    (\zaff{Lagrange/Sylow theorems in~\cite{DummitFoote}})
    However, any such group or order $2^k$ is isomorphic to the direct
    product of $k$ copies of $C_2$,
    where $C_2$ is the cyclic group of order $2$, \[ C_2 = \parens{ \set{0, 1}, 0, + } \].
    For instance, the Klein-Four group, $V_4$, is an abelian group of order $4$
    where every non-identity element has order $2$.
    We can construct an isomorphism from $V_4$ to $C_2 \times C_2$ as follows:
    \begin{align*}
      V_4 &= \parens{ \set{1, a, b, ab}, 1, \cdot } \\
      C_2 \times c_2 &= \parens{ \set{(0, 0), (0, 1), (1, 0), (1, 1)}, (0, 0), + }\\ \\
      \psi : V_4 &\to C_2 \times C_2 \\
      1 &\mapsto (0, 0) \\
      a &\mapsto (0, 1) \\
      b &\mapsto (1, 0) \\
      ab &\mapsto (1, 1) \\
    \end{align*}
    Therefore, since both $G$ and $H$ are isomorphic to a sequence of direct products of $C_2$
    with itself, so they are isomorphic to each other. 
  \end{proof}
\end{claim}

\begin{claim}~\label{claim:abelian-order-two-infinite-isomorphic}
  Any two groups $G$ and $H$ of infinite order such that every non-identity element
  has order $2$ are isomorphic.

  \begin{proof}
    First, note that $G$ and $H$ are each countably infinite sets since every element
    has order $2$.
    Construct a countably infinite group where every non-identity element has order $2$
    by taking the direct product of $C_2$ with itself an infinite number of times,
    \[
      C_2^{\infty} = C_2 \times C_2 \times C_2 \times \ldots 
    \]
    We can then construct isomorphisms from each of $G$ and $H$ to $C_2^{\infty}$,
    so $G$ and $H$ are also isomorphic to each other.
  \end{proof}
\end{claim}

Let $G$ be an abelian group in which every non-identity element has order $2$.
This means that $g = g = 0$, or equivalently $2g = 0$ for all $g \in G$.
As a result, $ng = 0$ for all $n \in 2\Z$ (multiples of $2$), so $G$ cannot have
a vector space structure over $Q$.
To define a vector space structure for $G$, we must limit the scalars to a
two-element field such as $\F_2$.
\[ \F_2 = \parens{\set{0, 1}, +, \times}. \]

\begin{claim}
  Any group $G$ in which every non-identity element has
  a vector space structure over the Galois field of two elements,
  $\F_2$.

  \begin{proof}
    Define multiplication as follows:
    \begin{align*}
      n g &= \begin{cases}
        0 & \text{if $n = 0$} \\
        g & \text{if $n = 1$} \\
      \end{cases} 
    \end{align*}
    Then all teh vector space axioms are satisfied. Particularly;
    \begin{enumarabic}
      \item $0g = 0$ for all $g \in G$.
      \item $1g = g$ for all $g \in G$.
      \item $a(bx) = (ab)x$ for any vector $x \in V$ and any two scalars $a, b \in F$.
        \begin{enumroman}
          \item $0(0g) = (0 \cdot 0)g = 0g = 0$.
          \item $1(0g) = (1 \cdot 0)g = 0g = 0$.
          \item $0(1g) = (0 \cdot 1)g = 0g = 0$.
          \item $1(1g) = (1 \cdot 1)g = 1g = g$.
        \end{enumroman}
      \item $(a+b)g = ag + bg$ for any vector $xg \in G$ and any two scalars $a, b \in F$.
        \begin{enumroman}
          \item $(0+0)g = 0g + 0g = 0$.
          \item $(0+1)g = 0g + 1g = 0 + g = g$.
          \item $(1+0)g = 1g + 0g = g$.
          \item $(1+1)g = 1g + 1g = 1g + 1g = 2g = 0$.
        \end{enumroman}
      \item $a(g+h) = ag + ah$ for any two vectors $x, y \in V$ and any scalar $a \in F$.
        \begin{enumroman}
          \item $1(g+h) = 1g + 1h = g + h$.
          \item $0(g+h) = 0g + 0h = 0$.
        \end{enumroman}
    \end{enumarabic}
  \end{proof}
\end{claim}












\newpage
\section{Axiomatizing Groups}~\label{sec:axiomatizing-abelian-groups}
In the previous section, we derived two vector space structures over two distinct fields,
$\Q$ and $\F_2$. In this section, we will officially model those two structures
using first-order logic.
First, we specify the following language:
\begin{align}
  \cL &= \vector{ 0, +, -, = }
\end{align}
The language dictates which symbols are permitted in well-formed axioms
for our structures.

\subsection{Axioms Defining Abelian Groups}~\label{sec:axioms-abelian-groups}

For any structure to be an abelian group, it needs to satisfy the group axioms
(closure, associativity, identity, and inversion) and commutativity
(see section~\ref{sec:def-groups} for reference).
We define the set $\Sigma$ to contain the following abelian group axioms:
\begin{align}
  \forall g\, \forall h\, &(g + h = h + g) &\text{ (commutativity)} \\
  \forall g\, \forall h\, \forall i\, &( (g + (h + i)) = ((g + h) + i) )
    &\text{ (associativity)} \\
  \forall g &( g + 0 = g) &\text{ ($0$ is the group identity element)} \\
  \forall g\, \exists h\, &(g + h = 0) &\text{ (group inversion)}
\end{align}
In first-order logic, $n$-ary functions are always defined to map elements
from $\abs{\fA}^n$ back to $\abs{\fA}$, so we do not need an explicit axiom
for closure.

\emph{Any structure satisfying $\Sigma$ is an abelian group.}


\subsection{Axioms Defining Divisible, Torsion Free Abelian Groups}~\label{sec:axioms-div-torsion-free}
Any abelian group needs additional restrictions to be divisible and torsion free.

\begin{enumarabic}
  \item Recall that a group $G$ is \emph{divisible} if for any $g \in G$
    and any $n \in \N$ with $n > 0$, there exists an element $h \in G$
    such that $n h = g$.
    We represent this condition using an infinite set of axioms
    specifying that, for each possible positive integer,
    there exists such sn element $h$ for each element $g$ in the structure:
    \[ T_1 = \set{ \forall g\, \exists h,\, (g = nh) \mid n = 1, 2, 3, \ldots }. \]
    
  \item Recall that a group $G$ is \emph{torsion free} if every non-identity element
    is torsion free. That is, if $g \neq 0$ then $ng \neq 0$ for any positive
    integer $n > 0$.
    We can represent this condition using an infinite set of elements
    specifying that, for each possible positive integer $n$,
    the product $ng$ is non-zero for each non-zero element in $G$:
    \[ T_2 = \set{ \forall g\, (g \neq 0 \lto ng \neq 0) \mid n = 1, 2, 3, \ldots }. \]
    
\end{enumarabic}

Let $T = T_1 \cup T_2$, then any model satisfying $\Sigma \cup T$
is a divisible, torsion free abelian group.



\subsection{Axioms Defining Abelian Groups Wherein Each Element Other Than the Identity
Has Order 2}~\label{sec:axioms-order-two-elements}

To specify that each element in an abelian group has order $2$,
define the set
\[ S = \set{\forall g\, ( g \neq 0 \lto (g + g = 0) ) }, \]
then any model satisfying $\Sigma \cup S$ is an abelian group
wherein each element other than the identity has order $2$.


% \newpage
\section{Categorical Analysis of the Structures}~\label{sec:categorical-analysis}
In this section, we will define some properties of the specific sets of axioms
based on what we have established about the corresponding structures.

\begin{claim}
  The set $(\Sigma \cup T)$ is not countably categorical.

  \begin{proof}
    By definition of $\Sigma$ and $T$, any structure satisfying $\Sigma \cup T$
    is a divisible, torsion free abelian group.
    Let $\fA$ be such a structure.
    As we established in the previous section, $\fA$ has a vector space structure over $\Q$.
    Since all finite-dimensional vector spaces over $\Q$ are countable and
    no two vector spaces of different dimensions can be isomorphic,
    there are more multiple countable models of $\Sigma \cup T$
    so $\Sigma \cup T$ is not countably categorical.
  \end{proof}
\end{claim}
\begin{claim}
  The set $(\Sigma \cup T)$ is categorical in the cardinality of $\R$.

  \begin{proof}
    Suppose $\fA$ and $\fB$ are two models of $\Sigma \cup T$,
    each having the cardinality of $\R$.
    Then \[ \dim(\fA) = \dim(\fB) = \abs{\R} \].
    Since any two vector spaces of the same dimension are isomorphic,
    $\fA$ and $\fB$ are isomorphic as vector spaces, hence
    isomorphic as groups.
    Therefore, $\Sigma \cup T$ has a single model in the cardinality $\R$
    up to isomorphism, hence is categorical in the cardinality of $\R$.
  \end{proof}
\end{claim}

\begin{claim}
  The set $(\Sigma \cup S)$ is countably categorical.

  \begin{proof}
    By definition of $\Sigma$ and $S$, any structure satisfying $\Sigma \cup S$
    is an abelian group wherein each element other than the identity has order $2$.
    Let $\fB$ be such a structure.
    Then $\fB$ has a vector space structure over $\F_2$.
    As shown in claim~\ref{claim:abelian-order-two-finite-isomorphic},
    any two such structures of the same finite order are isomorphic.
    Likewise, as shown in claim~\ref{claim:abelian-order-two-infinite-isomorphic},
    any two such structures of infinite order are countable, therefore isomorphic.
    Thus, for any fixed $n$, if there exists a model of $\Sigma \cup S$ of order $n$,
    then it is unique up to isomorphism, so $\Sigma \cup S$ is countably categorical.
  \end{proof}
\end{claim}




\newpage
\section{Conclusions}~\label{sec:conclusions}

\begin{claim}
  $\Cn \Sigma$ is not complete.

  \begin{proof}
    For an easy example, consider the fact that any model for $\Sigma \cup T$
    is a model for $\Sigma$ (or any divisible, torsion free abelian group is an abelian group).
    Now, there are sentences in $\cL$ that are true in $\Mod (\Sigma \cup T)$
    but are not in $\Cn \Sigma$. For instance, take the sentence saying that
    no element in a model of $\Sigma \cup T$ has order $2$:
    \[ \forall g\, (g \neq 0 \lto (g + g \neq 0)). \]
    This sentence is not deducible from $\Sigma$ --- and neither
    is its negation. In fact, it is entirely okay for elements in a model of $\Sigma$
    to have order $2$! Take $C_2$, the cyclic group of order $2$,
    for example, wherein the element $1$ has order $2$.
  \end{proof}

\end{claim}

\begin{claim}
  $\Cn (\Sigma \cup T)$ is complete.

  \begin{proof}
    As we established in the previous section, $\Sigma \cup T$ has no finite models.
    We also established that $\Sigma \cup T$ is categorical in the cardinality of the
    reals, but not countably categorical.
    First, note that $\Th \Mod (\Sigma \cup T) = \Cn (\Sigma \cup T)$,
    since $\Th \Mod (\Sigma \cup T)$ is the set of all sentences true in all models
    of $\Sigma \cup T$, and each sentence true in all models of $\Sigma \cup T$
    is in $\Cn (\Sigma \cup T)$.

    \step
    By Los-Vaught Test (cite)
    \begin{quotation}
      \crim{
        If a theory with no finite models is $\kappa$-categorical
        for some infinite cardinal $\kappa$, then the model is complete.
      }
    \end{quotation}
    \zaff{
      I found this result in book~\cite{ModelTheory}, am I allowed to use it
      (with proper citation), or should I be deriving this property myself?
    }
  \end{proof}
\end{claim}

\begin{corollary}
  $\Cn (\Sigma \cup T)$ is decidable.

  \begin{proof}
    We've seen that $\Cn (\Sigma \cup T) = \Th \Mod (\Sigma \cup T)$.
    This means that for any sentence $\sigma$ definable in $\cL$,
    if $\Cn (\Sigma \cup T) \models \sigma$ then 
    $\sigma \in \Cn (\Sigma \cup T)$.
    
    Furthermore, $\Cn (\Sigma \cup T)$ is complete,
    so for any sentence $\sigma$ definable in $\cL$,
    either $\Cn (\Sigma \cup T) \models \sigma$ or
    $\Cn (\Sigma \cup T) \models \neg \sigma$.

    Given an arbitrary sentence $\phi$, we can determine if $\phi \in \Cn (\Sigma \cup T)$
    by checking if $\Cn (\Sigma \cup T) \models \phi$ or $\Cn (\Sigma \cup T) \models \neg \phi$.
    In the first case, $\phi \in \Cn (\Sigma \cup T)$, and in the second case,
    $\phi \notin \Cn (\Sigma \cup T)$.
  \end{proof}
\end{corollary}

\newpage
\begin{claim}
  $\Cn \Sigma \cup S$ is not complete.

  \begin{proof}
    As shown in claim~\ref{claim:abelian-order-two-finite-isomorphic},
    every finite model of order $2^k$ for some integer $k > 0$ is isomorphic to
    the direct product of $k$ copies of $C_2$.
    Now, take any two finite models $\fA$ and $\fB$ of different orders, say $2^a$ and $2^b$.
    Define the sentence $\sigma_a$ to say that the structure has exactly $2^a$ elements,
    and the sentence $\sigma_b$ to say that the structure has exactly $2^b$ elements.
    \begin{align*}
      \sigma_a &= \forall g_1 \exists g_2 \exists \ldots \exists g_{2^a}
      ((g_1 \neq g_2) \land (g_1 \neq g_3) \land \ldots \land (g_2 \neq g_3) \land(g_2 \neq g_4) \land \ldots \land g_{2^a-1} \neq (g_{2^a})) \\
      \sigma_b &= \forall g_1 \exists g_2 \exists \ldots \exists g_{2^b}
      ((g_1 \neq g_2) \land (g_1 \neq g_3) \land \ldots \land (g_2 \neq g_3) \land(g_2 \neq g_4) \land \ldots \land g_{2^b-1} \neq (g_{2^b})) \\
    \end{align*}
    Then, $\sigma_a$ is true in $\fA$ but not in $\fB$, and $\sigma_b$ is true in $\fB$ but not in $\fA$.
    However;
    \begin{align*}
      \Cn (\Sigma \cup S) &\not\models \sigma_a \\
      \Cn (\Sigma \cup S) &\not\models (\neg \sigma_a) \\
      \Cn (\Sigma \cup S) &\not\models \sigma_b \\
      \Cn (\Sigma \cup S) &\not\models (\neg \sigma_b)
    \end{align*}
    Consequently, $\Cn \Sigma \cup S$ is not complete.
  \end{proof}
\end{claim}

For $\Cn \Sigma \cup S$, we must specify that no model is finite,
thus limiting the cardinality to infinite models.

To do this, we define a new set $R$ specifying that, for each possible finite cardinality
$n$, there exists more than $n$ elements in the structure.


Let $R$ be the set of all sentences of the form
\[ \set{ \exists g_1\, \exists g_2, \exists \ldots \exists g_n
((g_1 \neq g_2) \land (g_1 \neq g_3) \land \ldots \land (g_1 \neq g_n) \land (g_2 \neq g_3)
\land \ldots 
\land (g_2 \neq g_n) \land \ldots \land )} \]

\[
  R = \set{ \exists g_1\, \exists g_2\, \exists \ldots \exists g_n
    \bigwedge_{i=1}^{n-1} \bigwedge_{j=i+1}^{n} (g_i \neq g_j)
    \mid n = 1, 2, 3, \ldots }
\]


\bigskip
\crim{
  Still working on other conclusions and Item $5$.
}

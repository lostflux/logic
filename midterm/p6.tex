\begin{problem}
  This is a problem in first order logic.

  \step
  It is also a short answer problem.
  That is, giving a correct answer is sufficient;
  no explanation is needed.

  \step
  Feel free to use the following abbreviations and conventions:
  You may use any of our five sentential logic connectives,
  write $=, <, +, -$ using infix notation
  (e.g. $x = y$ instead of $= xy$), use the quantifier $\exists$,
  call your variables $x$ and $y$ (or even $\delta$ and $\eps$),
  and omit parenthesis, include extra parentheses,
  and use other kinds or sizes of parentheses
  (such as $[ ]$ and ${}$) to enhance readability.

  \step
  However, ``$(\forall \eps > 0)$,'' ``$\le$,'' and ``$\ne$'',
  for example, are not abbreviations we have defined.

  \step
  You may, however, define your own abbreviations.
  For example, you may say something like:
  ``for any terms $t_1$ and $t_2$, let $t_1 > t_2$ be an abbreviation
  for the formula $t_2 < t_1$.

  \step
  If you do this, be careful to be technically correct.
  For example, do noart say
  ``Let $\le$ be an abbreviation for $< \lor =$''.
  
%   \step
%   Let $\frakL$ be the language for first order logic that has equality,
%   constant symbols $0$ and $r$, one-place function symbols $f$ and $a$,
%   two-place function symbols $+$, $−$, and $\cdot$,
%   and a two-place predicate symbol $<$. Translate the
% parameters of this language as follows. Note: “Translate r as a real number”
% means that r is the name of some specific real number, we just aren’t saying
% which one; similarly for “translate f as a function.”
\end{problem}

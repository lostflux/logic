\begin{problem}
  This is a problem in sentential logic.

  \step
  Let $v$ be a truth assignment on the set of sentence symbols.
  For any wff $\alpha$, let;
  \begin{enumroman}
    \item $f(\alpha)$ denote the number of occurrences of
      $\iff$ in $\alpha$.
    \item $g(\alpha)$ denote the number of occurrences in $\alpha$
      of sentence symbols $A_i$ for which $v(A_i) = T$.
  \end{enumroman}
  For example, if $v(A_1) = T$ and $v(A_2) = F$,
  and $\alpha$ is $((A_1 \land A_2) \lto \lnot A_1)$,
  then $g(\alpha) = 2$.
  \begin{enumalph}
    \item Give careful definitions of $f(\alpha)$ and $g(\alpha)$
      by recursion on $\alpha$.
      \begin{Answer}
        \begin{align*}
          f(\alpha) &=
            \begin{cases}
              0 & \text{if $\alpha$ is a sentence symbol} \\
              f(\alpha_1) & \text{if $\alpha$ is $(\lnot \alpha_1$) for some wff $\alpha_1$} \\
              f(\alpha_1) + f(\alpha_2) + 1 & \text{if $\alpha$ is $(\alpha_1 \iff \alpha_2)$ for some wffs $\alpha_1$ and $\alpha_2$} \\
              f(\alpha_1) + f(\alpha_2) & \text{if $\alpha$ is $(\alpha_1 \ast \alpha_2)$, where $\ast$ is a binary logical connective.} \\
            \end{cases} \\
          g(\alpha) &=
            \begin{cases}
              0 & \text{if $\alpha$ is a sentence symbol and $v(\alpha) = F$} \\
              1 & \text{if $\alpha$ is a sentence symbol and $v(\alpha) = T$} \\
              g(\alpha_1) & \text{if $\alpha$ is $(\lnot \alpha_1)$ for some wff $\alpha_1$} \\
              g(\alpha_1) + g(\alpha_2) &
                \text{if $\alpha$ is $(\alpha_1 \ast \alpha_2)$, where $\ast$ is a binary logical connective.}\\
            \end{cases}
        \end{align*}
      \end{Answer}
    \newpage
    \item Prove (carefully and formally) by induction on $\alpha$
      that if $\iff$ is the only connective symbol occurring in $\alpha$,
      then $\vbar(\alpha) = T$ if and only if $f(\alpha)$ and $g(\alpha)$
      have the opposite parity (i.e., one is even and the other is odd).

      \step
      \crim{(See next page.)}

      \newpage
      \begin{claim}
        If $\alpha$ is a wff with no other connective symbols than $\iff$,
        then $\vbar(\alpha) = T$ if and only if $f(\alpha)$ and $g(\alpha)$
        have the opposite parity.

        \begin{proof}
          
          Let $\alpha$ be a wff with no other connective symbols than $\iff$.
          We shall prove by induction on the structure of $\alpha$.
    
          \step
          \textbf{Base Case:} If $\alpha$ is a sentence symbol, then $f(\alpha) = 0$.
            There are two cases:
            \begin{enumroman}
              \item If $v(\alpha) = T$, then $g(\alpha) = 1$, so $f(\alpha)$ and $g(\alpha)$
                have opposing parity and the claim holds.
              \item If $v(\alpha) = F$, then $g(\alpha) = 0$, so $f(\alpha)$ and $g(\alpha)$
                have the same parity and the claim holds.
            \end{enumroman}
    
          \step
          \textbf{Inductive Step:}
          If $\alpha$ is $(\alpha_1 \iff \alpha_2)$
          for some wffs $\alpha_1$ and $\alpha_2$ and the claim is true for 
          $\alpha_1$ and $\alpha_2$, we show that the claim is also true for $\alpha$.
          There are $16$ possibilities:
          \begin{table}[H]
            \begin{tabular}{|c c c | c c c | c c c|}
              \bottomrule
              $f(\alpha_1)$ & $g(\alpha_1)$ & $\vbar(\alpha_1)$ &
              $f(\alpha_2)$ & $g(\alpha_2)$ & $\vbar(\alpha_2)$ &
              $f(\alpha_1 \iff \alpha_2)$ & $g(\alpha_1 \iff \alpha_2)$ & $\vbar(\alpha_1 \iff \alpha_2)$\\
              \midrule
              \green{even} & \green{even} & $F$ & \green{even} & \green{even} & $F$ & \crim{odd} & \green{even} & $T$ \\
              \green{even} & \green{even} & $F$ & \green{even} & \crim{odd} & $T$ & \crim{odd} & \crim{odd} & $F$ \\
              \green{even} & \green{even} & $F$ & \crim{odd} & \green{even} & $T$ & \green{even} & \green{even} & $F$ \\
              \green{even} & \green{even} & $F$ & \crim{odd} & \crim{odd} & $F$ & \green{even} & \crim{odd} & $T$ \\
              \green{even} & \crim{odd} & $T$ & \green{even} & \green{even} & $F$ & \crim{odd} & \crim{odd} & $F$ \\
              \green{even} & \crim{odd} & $T$ & \green{even} & \crim{odd} & $T$ & \crim{odd} & \green{even} & $T$ \\
              \green{even} & \crim{odd} & $T$ & \crim{odd} & \green{even} & $T$ & \green{even} & \crim{odd} & $T$ \\
              \green{even} & \crim{odd} & $T$ & \crim{odd} & \crim{odd} & $F$ & \green{even} & \green{even} & $F$ \\
              \crim{odd} & \green{even} & $T$ & \green{even} & \green{even} & $F$ & \green{even} & \green{even} & $F$ \\
              \crim{odd} & \green{even} & $T$ & \green{even} & \crim{odd} & $T$ & \green{even} & \crim{odd} & $T$ \\
              \crim{odd} & \green{even} & $T$ & \crim{odd} & \green{even} & $T$ & \crim{odd} & \green{even} & $T$ \\
              \crim{odd} & \green{even} & $T$ & \crim{odd} & \crim{odd} & $F$ & \crim{odd} & \crim{odd} & $F$ \\
              \crim{odd} & \crim{odd} & $F$ & \green{even} & \green{even} & $F$ & \green{even} & \crim{odd} & $T$ \\
              \crim{odd} & \crim{odd} & $F$ & \green{even} & \crim{odd} & $T$ & \green{even} & \green{even} & $F$ \\
              \crim{odd} & \crim{odd} & $F$ & \crim{odd} & \green{even} & $T$ & \crim{odd} & \crim{odd} & $F$ \\
              \crim{odd} & \crim{odd} & $F$ & \crim{odd} & \crim{odd} & $F$ & \crim{odd} & \green{even} & $T$ \\
              \toprule
            \end{tabular}
          \end{table}

          \step
          As we can infer from the table, 
          $f(\alpha_1 \iff \alpha_2)$ and $g(\alpha_1 \iff \alpha_2)$
          always have opposing parity whenever $\vbar(\alpha_1 \iff \alpha_2) = T$.
        \end{proof}
      \end{claim}
  \end{enumalph}
\end{problem}

\begin{problem}
  This is a problem in sentential logic.

  \step
  Suppose that $\alpha$, $\beta$, and $\gamma$ are wffs
  such that $\alpha \models \gamma$ and $\beta \models \lnot \gamma$.
  Show that $\alpha \models \lnot \beta$.

  \step
  \emph{
    Please do this formally, by showing that every truth assignment that
    satisfies $\alpha$ also satisfies $(\lnot \beta)$.
    You will be graded on whether you have a
    correct proof of this kind.
  }
\end{problem}
\begin{Answer}
  Let $\alpha$ be a wff and $v$ be a truth assignment satisfying $\alpha$.
  such that  $\vbar(\alpha) = T$.
  Since $\alpha \models \gamma$, we also have that $\vbar(\gamma) = T$.

  \step
  Consider the truth assignment of $v$ on $\beta$.
  There are two possible assignments; either $T$ or $F$.
  Suppose $\vbar(\beta) = T$. Since $\beta \models \lnot \gamma$,
  this implies that $\vbar(\lnot \gamma) = T$, so $\vbar(\gamma) = T$,
  contradicting the deduction of $\gamma$ from the assignment $\vbar(\alpha) = T$.
  Therefore, whenever a truth assignment $v$ satisfies $\alpha$,
  it must also assign the value $F$ to $\beta$, i.e. it must satisfy $(\lnot \beta)$
  in order to avoid a contradicting deduction of $\gamma$.
\end{Answer}


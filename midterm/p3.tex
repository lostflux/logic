\begin{problem}
  This is a problem in sentential logic.

  \step
  Show that the Compactness Theorem can be proven from 
  the Soundness Theorem and the Completeness Theorem.

  \step
  \textbf{Compactness Theorem:} If $\Sigma$ is an infinite set of wffs,
  then $\Sigma$ is satisfiable if $\Sigma$ is finitely satisfiable.

  \step
  \textbf{Completeness Theorem:} If $\Gamma \models \alpha$ then $\Gamma \deduces \alpha$.

  \step
  \textbf{Soundness Theorem:} If $\Gamma \deduces \alpha$ then $\Gamma \models \alpha$. 
\end{problem}
\begin{Answer}
  Let $\Sigma$ be an infinite set of wffs.\\
  We aim to show that if $\Sigma$ is finitely satisfiable then it is satisfiable.
  An easier route is to prove the contrapositive: if $\Sigma$ is not satisfiable,
  then it is not finitely satisfiable.

  \step
  Suppose $\Sigma$ is not satisfiable, then $\Sigma \models (\sigma \land \lnot \sigma)$
  for some wff $\sigma$. By the completeness theorem, $\Sigma \deduces (\sigma \land \lnot \sigma)$.
  This implies that there exists a finite
  $\Sigma_0 \subset \Sigma$, such that $\Sigma_0 \deduces (\sigma \land \lnot \sigma)$.
  By the soundness theorem, we have that $\Sigma_0 \models (\sigma \land \lnot \sigma)$.
  Therefore, $\Sigma_0$ is not satisfiable, hence $\Sigma$ cannot be finitely satisfiable
  since $\Sigma_0$ is a subset of $\Sigma$.
\end{Answer}

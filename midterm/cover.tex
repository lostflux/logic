

% TITLE
\newdate{due-date}{30}{01}{2023}
\exam{1 --- \displaydate{due-date}}{Winter `23}{Marcia Groszek}{Amittai Siavava}{Math 69: Logic}

%CREDIT STATEMENT
% \CreditStatement{
%   I worked on these problems alone,
%   with reference to class notes and the following books:
%   \begin{enumerate}
%     \item \textbf{A Mathematical Introduction to Logic} by \textbf {Herbert Enderton}.
%   \end{enumerate}
% }

\bigskip
You may consult your textbook, notes, class handouts, and returned
homework as you work on this exam, but you should not discuss the exam
with anybody other than the professor, or look in other textbooks or on the
internet (except on the course web page).

It is still okay to discuss class worksheets and homework problems with
each other, even if they are related to exam problems, as long as you do not
discuss any possible relevance to the exam.

Please ask the professor if you have any questions about the exam.
You can use anything from the portions of the text we have covered,
including the results of homework problems that were assigned for graded
homework. (If you want to use the result of a homework problem that wasn't
assigned, you must first solve the problem, and include the solution in your
answer.) You can use material from class handouts and worksheets, including
the results of problems. You can also use earlier parts of an exam problem in
the solutions to later parts of that same problem, even if you were not able
to solve the earlier parts.

Your exam paper should follow the following format rules: Identify each
problem by number, and also repeat or restate the problem before giving a
solution.

The exam will be graded on the clarity and completeness of your explanations,
and the correct use of mathematical notation and terminology, as
well as on the content of your answers.

\title{\Huge{Math 71: Algebra\\Groups of Small Order}}
\author{\Large{Amittai Siavava}}
\date{\Large{\today}}
% \documentclass[11pt, reqno]{amsart}
\documentclass[english, 11pt, a4paper, twoside, openright, oldfontcommands]{amsart}

\input{../../common/macros.tex}
\usepackage[backend=bibtex,style=science]{biblatex}
\geometry{top=0.75in,left=2in,bottom=0.75in,right=0.80in}
\bibliography{main.bib}
\pgfplotsset{compat=1.18}

\pagestyle{fancy}                         % fancy (allow headers, footers)
\fancyhf{}                                % clear all header/footer settings.
\cfoot{\thepage}                          % set page-numbers in footer.
% \lhead{\textit{\textbf{ Amittai, S}}}   % set name in header, left.
% \rhead{\textsc{Math 71: Algebra}}       % set class name in header, right.
\renewcommand{\headrulewidth}{0pt}
\renewcommand{\footrulewidth}{0pt}

\addtolength{\oddsidemargin}{-.45in}
\addtolength{\evensidemargin}{.45in}


\newcounter{problem}
\setcounter{problem}{0}

\renewcommand{\theenumi}{\alph{enumi}}

\begin{document}

\setlength{\headheight}{13.0pt}
\setlength{\footskip}{15.0pt}

\flushleft{\textbf{Comments}}

I also  read through the part on recursions and
I thought the recursive definitions of well-formed formulas
as $n$-ary functions taking in $n$
arguments (each of which may be a function applied to its arguments)
made a lot of sense.

On the other hand, polish notation without parenthesizing
the arguments felt more complicated than the infix notation we
have been using thus far. I understand how it reduces ambiguity,
but I found it more tasking to process and mentally parse the wffs written in polish notation.

\flushleft{\textbf{Questions}}

In the section about $0$-ary connectives,
we say that $\top$ and $\bot$ can be thought of as
the constants $T$ and $F$ having $\overline{v}(\bot) = F$ and $\overline{v}(\top) = T$
for every $v$.
Wouldn't we get a contradiction if we have a function that negates $v$?
For instance, if $w = \lnot v$ for some function $v$,
doesn't $\overline{w}(\bot) = \lnot \overline{v}(\bot)$ imply that
$\overline{w}(\bot) = T$?

\flushleft{\textbf{Exercises}}\\
\begin{problem}
  Let G be the following three-place Boolean function.
  \begin{align*}
    G(F, F, F) &= T,&
    G(T, F, F) &= T,\\
    G(F, F, T) &= T,&
    G(T, F, T) &= F,\\
    G(F, T, F) &= T,&
    G(T, T, F) &= F,\\
    G(F, T, T) &= F,&
    G(T, T, T) &= F.
  \end{align*}
  \begin{enumalph}
    \item Find a wff using at most the connectives $\land, \lor$, and $\lnot$,
      that realizes $G$.
      \begin{Answer}
        Truth table for $G$:

        \begin{tabular}{ c c c c }
          $\alpha$ & $\beta$ & $\gamma$ & $G(\alpha, \beta, \gamma)$ \\
          \bottomrule
          $T$ & $T$ & $T$ & $F$ \\
          \midrule
          $T$ & $T$ & $F$ & $F$ \\
          \midrule
          $T$ & $F$ & $T$ & $F$ \\
          \midrule
          $T$ & $F$ & $F$ & $T$ \\
          \midrule
          $F$ & $T$ & $T$ & $F$ \\
          \midrule
          $F$ & $T$ & $F$ & $T$ \\
          \midrule
          $F$ & $F$ & $T$ & $T$ \\
          \midrule
          $F$ & $F$ & $F$ & $T$ \\
          \toprule
        \end{tabular}
        \\ \\
        $G(\alpha, \beta, \gamma)$ always disagrees with the majority
        of $\alpha$, $\beta$, and $\gamma$.
        
        \noindent
        That is, 
        \[ 
          G(\alpha, \beta, \gamma ) = 
            \lnot (
              (\alpha \land \beta ) \lor (\alpha \land \gamma )
              \lor (\beta \land \gamma )
            )
        \]

      \end{Answer}
    \item Then find such a wff in which connective symbols occur at not more than $5$ places.
      \begin{Answer}
        \[ G(\alpha, \beta, \gamma) = \lnot(\# \alpha \beta \gamma)\]

        Using only $\land, \lor, \lnot$:

        \[ G(\alpha, \beta, \gamma) = \lnot(\;(\alpha \land (\beta \lor \gamma))\; \lor \; (\beta \land \gamma))\]
      \end{Answer}
  \end{enumalph}
\end{problem}
% \end{vplace}
\bigskip
\end{document}

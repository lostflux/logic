% \documentclass[11pt, reqno]{amsart}
\documentclass[english, 12pt]{amsart}

\input{~/latex-common/macros.tex}
% \usepackage[backend=bibtex,style=science]{biblatex}
\pgfplotsset{compat=1.18}

\pagestyle{fancy}                         % fancy (allow headers, footers)
\fancyhf{}                                % clear all header/footer settings.
\cfoot{\thepage}                          % set page-numbers in footer.
\renewcommand{\headrulewidth}{0pt}
\renewcommand{\footrulewidth}{0pt}
\begin{document}

\setlength{\headheight}{13.0pt}
\setlength{\footskip}{15.0pt}


% TITLE
\reading{February 01, 2023}{Winter `23}{Marcia Groszek}{Amittai Siavava}{Math 69: Logic}


\begin{problem}
  Suppose a formal deduction includes the line $Px$.
  Can the next line be $(Px \lor Qx)$, with
  ``Rule T'' as the justification?
  \begin{Answer}
    Yes, I think it is possible to have $(Px \lor Qx)$
    as the next line with ``Rule T''
    since $Px$ tautologically implies $Px \lor Qx$.
  \end{Answer}
\end{problem}

\step
% \begin{blockcolor}
  \begin{center}\textbf{Questions} \end{center}

  \step
  I had no questions about the reading,
  but reviewing alphabetic variants would be helpful.
\bigskip
\end{document}

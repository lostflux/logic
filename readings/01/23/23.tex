% \documentclass[11pt, reqno]{amsart}
\documentclass[english, 12pt]{amsart}

\input{../../../common/macros.tex}
\usepackage[backend=bibtex,style=science]{biblatex}
\bibliography{main.bib}
\pgfplotsset{compat=1.18}

\pagestyle{fancy}                         % fancy (allow headers, footers)
\fancyhf{}                                % clear all header/footer settings.
\cfoot{\thepage}                          % set page-numbers in footer.
\renewcommand{\headrulewidth}{0pt}
\renewcommand{\footrulewidth}{0pt}
\begin{document}

\setlength{\headheight}{13.0pt}
\setlength{\footskip}{15.0pt}


% TITLE
\reading{February 01, 2023}{Winter `23}{Marcia Groszek}{Amittai Siavava}{Math 69: Logic}


\begin{problem}
  Suppose that $x$ and $y$ are variables, and our language
  includes the symbol $=$ and constant symbols $c$ and $d$.
  Show the sentence $c = d \to d = c$ is satisfied by every
  structure (for this language) and every variable assignment.

  \begin{Answer}
    Let $v$ be a variable assignment.
    There are two possibilities for $\vbar(c=d)$:
    \begin{enumroman}
      \item $\vbar(c=d) = T$
        This implies that $v(c) = v(d)$, so
        it follows that $v(d) = v(c)$
        and $\vbar(d=c) = T$.
        Therefore, $\vbar(c=d \to d=c) = T$.
      \item $\vbar(c=d) = F$
        This implies that $v(c) \neq v(d)$.
        However, our sentence now reads
        $\vbar(F \to (d=c))$,
        which is satisfied.
    \end{enumroman}
  \end{Answer}
\end{problem}

\step
% \begin{blockcolor}
  \begin{center}\textbf{Questions} \end{center}

  \step
  I was somewhat confused when the textbook talks about
  ``the translation of $\phi$ determined by $\frakA$''
  (page 83).
  Is it akin to the equivalence of $\phi$ in $\frakA$,
  or is there additional nuance that I am missing?

\bigskip
\end{document}

% \documentclass[11pt, reqno]{amsart}
\documentclass[english, 11pt]{amsart}

\input{../../../common/macros.tex}
\usepackage[backend=bibtex,style=science]{biblatex}
\bibliography{main.bib}
\pgfplotsset{compat=1.18}

\pagestyle{fancy}                         % fancy (allow headers, footers)
\fancyhf{}                                % clear all header/footer settings.
\cfoot{\thepage}                          % set page-numbers in footer.
\renewcommand{\headrulewidth}{0pt}
\renewcommand{\footrulewidth}{0pt}
\begin{document}

\setlength{\headheight}{13.0pt}
\setlength{\footskip}{15.0pt}


% TITLE
\reading{February 01, 2023}{Winter `23}{Marcia Groszek}{Amittai Siavava}{Math 69: Logic}

  \begin{problem}
    Assume that we have a language with the following parameters:
    \begin{itemize}
      \item $\forall$, intended to mean ``for all things'';
      \item $N$, intended to mean ``is a number'';
      \item $I$, intended to mean ``is interesting'';
      \item $<$, intended to mean ``is less than'';
      \item and $0$, a constant symbol intended to denote zero.
    \end{itemize}
  Translate into this language the English sentences listed below.
  
  \step
  \emph{If the English sentence is ambiguous,
        you will need more than one translation.}

  \begin{enumalph}
    \setcounter{enumi}{1}
    % \item Zero is less than any number.
    % \begin{Answer}
    %   \[
    %     \forall x \left( Nx \to (0 < x) \right)
    %   \]
    % \end{Answer}
    \item If any number is interesting, then zero is interesting.
    \begin{Answer}
      \[
        \forall x \left( Nx \to Ix \right) \to I0
      \]
    \end{Answer}
  \end{enumalph}
\end{problem}
    \step
    \begin{blockcolor}
      \begin{center}\textbf{Questions} \end{center}

      \step
      I am somewhat confused about free variables.
      I understand the example and recursive explanation of
      what free variables are ---
      but I do not understand the context around free variables.
      Is it ``free'' because it is singular (i.e. not declared in some way
      before use, like in $y$ in $\forall x\ x \in x y$)?
    \end{blockcolor}
\bigskip
\end{document}

% \documentclass[11pt, reqno]{amsart}
\documentclass[english, 12pt]{amsart}

\input{../../../common/macros.tex}
\usepackage[backend=bibtex,style=science]{biblatex}
\bibliography{main.bib}
\pgfplotsset{compat=1.18}

\pagestyle{fancy}                         % fancy (allow headers, footers)
\fancyhf{}                                % clear all header/footer settings.
\cfoot{\thepage}                          % set page-numbers in footer.
\renewcommand{\headrulewidth}{0pt}
\renewcommand{\footrulewidth}{0pt}
\begin{document}

\setlength{\headheight}{13.0pt}
\setlength{\footskip}{15.0pt}


% TITLE
\reading{February 01, 2023}{Winter `23}{Marcia Groszek}{Amittai Siavava}{Math 69: Logic}


\begin{problem}
  Suppose our language contains the equality symbol,
  constant symbol $0$, and no other parameters,
  and our structure $\frakA$ is defined by setting $\abs{\frakA} = \N$,
  and $c^\frakA = 0$.
  Which sets of natural numbers are definable in this structure?

  \begin{Answer}
    By definition;
    $\abs{\frakA} = \N$ and
    $c^\frakA = 0$.
    
    \step
    First, $0 \in \abs{\frakA}$ (since $\frakA$ assigns
    $c^\frakA$ to a member of $\abs{\frakA}$).

    \step
    We also have a two-place predicate symbol $=$,
    but we do not have a ``forall'' quantifier so we cannot
    enforce the predicate on every member of $\abs{\frakA}$.
    Therefore, all sets $S \subseteq \N$ such that $0 \in S$ are
    definable in the structure.
  \end{Answer}
\end{problem}

\step
% \begin{blockcolor}
  \begin{center}\textbf{Questions} \end{center}

  \step
  The textbook gives an example that, on the structure
  $(\R, <)$, a function $h$ is an automorphism if it is strictly increasing
  (i.e. $a < b \implies h(a) < h(b)$),
  e.g. $h(a) = a^3$,
  but that $\N$ is not definable in this structure since
  $h$ maps elements out of $\N$ into $\N$.
  I don't fully understand what this means.
\bigskip
\end{document}

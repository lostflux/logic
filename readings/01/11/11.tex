% \documentclass[11pt, reqno]{amsart}
\documentclass[english, 11pt]{amsart}

\input{../../../common/macros.tex}
\usepackage[backend=bibtex,style=science]{biblatex}
\bibliography{main.bib}
\pgfplotsset{compat=1.18}

\pagestyle{fancy}                         % fancy (allow headers, footers)
\fancyhf{}                                % clear all header/footer settings.
\cfoot{\thepage}                          % set page-numbers in footer.
\renewcommand{\headrulewidth}{0pt}
\renewcommand{\footrulewidth}{0pt}
\begin{document}

\setlength{\headheight}{13.0pt}
\setlength{\footskip}{15.0pt}


% TITLE
\reading{February 01, 2023}{Winter `23}{Marcia Groszek}{Amittai Siavava}{Math 69: Logic}

% \newpage
% \flushleft{\textbf{Exercises}}\\
  \begin{problem}
    Is the set of wffs that are not tautologies decidable?
    Explain briefly.
  \end{problem}
  \begin{Answer}
    Yes, the set is decidable.
    
    \step
    Since the set of all tautologies is decidable
    (Taking the set of all tautologies
    as the tautological implications of the empty set,
    then Theorem $17$C and Corollary $17$D tell us that
    the tautological implications of a finite set are decidable).
    Let $T$ be the list of all tautologies, then
    the set of all wffs that are not tautologies is
    \[
      U = \{ x : x \text{ is a wff and } x \notin T \}.  
    \]
    Since $T$ is decidable:
    \begin{itemize}
      \item We can always evaluate whether a wff is in $T$,
        thus determine if it is not in $U$.
      \item We can always evaluate whether a wff is not in $T$,
        thus determine if it is in $U$.
    \end{itemize}
    Therefore, $U$ is decidable.

    \[ \forall x\; \forall y\; (([x] = [y]) \lor ([x] \cap [y] = \varnothing))\]

  \end{Answer}
    \step
    \begin{blockcolor}
      \begin{center}\textbf{Questions} \end{center}

      \step
      If we had an infinite set $\Sigma$,
      then it may be semidecidable, then we may not
      be able to determine if a wff is not in the set.
      In this case, would it be correct to infer that the set of all wffs
      that are not tautological consequences of $\Sigma$
      is not decidable?

      \step
      In general, is it always true to say that the complement of a
      decidable set is decidable?
    \end{blockcolor}
\bigskip
\end{document}

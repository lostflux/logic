\title{\Huge{Math 71: Algebra\\Groups of Small Order}}
\author{\Large{Amittai Siavava}}
\date{\Large{\today}}
% \documentclass[11pt, reqno]{amsart}
\documentclass{memoir}

\input{../../common/macros.tex}
\usepackage[backend=bibtex,style=science]{biblatex}
\geometry{top=0.75in,left=2in,bottom=0.75in,right=0.80in}
\bibliography{main.bib}
\pgfplotsset{compat=1.18}

\pagestyle{fancy}                         % fancy (allow headers, footers)
\fancyhf{}                                % clear all header/footer settings.
\cfoot{\thepage}                          % set page-numbers in footer.
% \lhead{\textit{\textbf{ Amittai, S}}}   % set name in header, left.
% \rhead{\textsc{Math 71: Algebra}}       % set class name in header, right.
\renewcommand{\headrulewidth}{0pt}
\renewcommand{\footrulewidth}{0pt}

\addtolength{\oddsidemargin}{-.875in}
\addtolength{\evensidemargin}{-.875in}


\newcounter{problem}
\setcounter{problem}{0}

\renewcommand{\theenumi}{\alph{enumi}}

\begin{document}

\setlength{\headheight}{13.0pt}
\setlength{\footskip}{15.0pt}

\begin{vplace}[0.7]
  I am intrigued by the abstract formulation of `language'
  as a set of symbols and rules that allow us to communicate
  and reason about the world or specific aspects of it.
  One of my other classes (theory of computation) is also
  starting with an abstract look at language, albeit from
  the perspective of trying to understand computation
  and derive computing machines.
  
  From Wednesday's class,
  I was somewhat unsure why we deduced the equivalence
  of \crim{$A$ only if $B$} to \crim{$A \leftarrow B$}.
  I am `if' and `if and only if' but
  had never encountered `only if' on its own ---
  but after reading through the material again,
  I now understand the equivalence.

  On the assigned reading,
  Given any set of sentences $\tau$,
  we say that \crim{$\varnothing \models \tau$}
  (on page 23).
  Is it also logically valid to say that \crim{$\tau \models \varnothing$}?
  (Since if we have a valid set of rules
  and add no further restrictions,
  the rules remain valid)?
  Could we therefore say any set of rules $\tau$ is
  \emph{tautologically equivalent} to the empty set?
\end{vplace}
\bigskip
\end{document}

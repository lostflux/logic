\begin{problem}
  Let $\cL$ be the language with the equality symbol, a constant symbol $0$,
  and a two-place predicate symbol $<$.
  Let the structure $\fA = \vector{Z; 0,<}$ be the integers
  with distinguished element $0$ and the usual ordering;
  that is $\abs{\fA} = \Z = \set{\ldots,-3,-2,-1,0,1,2,3, \ldots}$,
  $0^\fA = 0$, and $<^\fA = \set{(m,n) \mid m < n}$. 

  \step
  Show there is a countable structure $\frakB$ for $\calL$
  that is elementarily equivalent to $\frakA$
  and has the following property:
  There is an element $a \in \abs{\mathfrak B}$ such that
  $\{ b \in \abs{\mathfrak B} \mid 0^\frakB <^\frakB b <^\frakB a \}$
  is infinite.

  \step
  For this problem, you do not have to be too formal about
  justifying claims like
  ``every model of $\sigma$ has property $\Phi$'' or
  ``$\frakC$ with variable assignment $s$ satisfies $\alpha$.''
  
  \step
  For example, if $\sigma$ is $\exists x \, \exists y \, x \neq y$,
  it is obvious that every model of $\sigma$ has size at least $2$
  and that $\frakA$ is a model of $\sigma$; you need not prove this.

  \begin{Answer}
    $\fA$ is a model of a discrete linear order without endpoints.

    \step
    Define $\fB$ such that $\abs{\fB} = \Z \times \set{0,1}$,
    $\fB$ has two distinguished elements $(0, 0)$ and $(0, 1)$,
    $0^\fB = (0, 0)$,
    and following the usual ordering of $\Z$ on the first coordinate
    of elements in $\abs{\fB}$, i.e. $\abs{\fB} = $
    \begin{align*}
      &\set{\ldots, (-3, 0), (-2, 0), (-1, 0), (0, 0), (0, 1), (1, 0), (2, 0), (3, 0), \ldots }
    \\ &\cup \\
    &\set{\ldots, (-3, 1), (-2, 1), (-1, 1), (0, 1), (1, 1), (2, 1), (3, 1), \ldots }
    \end{align*}
    Define $<^\fB = \set{((a_1, b_1), (a_2, b_2)) \mid
    (b_1 < b_2) \lor (\lnot (b_2 < b_1) \land (a_1 < a_2))}$.

    \step
    Then $\fB$ is countable and there exists an element
    $a \in \abs{\fB}$ such that the set
    $S = \set{ b \in \abs{\fB} \mid 0^\fB <^\fB b <^\fB a}$ is infinite
    (take $a = (0, 1)$, for example, then
    $S = \set{(n, 0) \mid n \in \Z, n > 0}$, which is infinite).

  \end{Answer}
\end{problem}

\begin{problem}

In each case, either show without using the Completeness Theorem that
$\phi \deduces \psi$, or else show that $\phi \not\models \psi$.
Here $P$ and $Q$ are one-place predicate symbols.

To show $\phi \deduces \psi$, you may use any of the metatheorems of Section 2.4.  You do not need to give an actual deduction.

For this problem, if you want to give an example of a structure and variable
assignment satisfying some particular formula $\gamma$,
it is enough to specify the structure and variable assignment.
You do not have to formally prove that $\gamma$ is satisfied.

\begin{enumroman}
  \item $\phi$ is $\forall x \, (Px \lor Qx)$, 
    and $\psi$ is $\forall x \, Px \lor \forall x \, Qx$.
    \begin{Answer}
      In this case, $\phi \not\models \psi$.

      \step
      Let $\fA$ be a structure with $\abs{\fA} = \set{p, q}$ such that
      $P^\fA = \set{p}$ and $Q^\fA = \set{q}$.
      Let $s$ be a variable assignment.\\
      Then the sentence $\forall x \, (Px \lor Qx)$ is satisfied by $s$ since
      $\models_{\fA}(Px \lor Qx)[s(x \mid p)]$ and
      $\models_{\fA}(Px \lor Qx)[s(x \mid q)]$.\\
      However, $\forall x \, Px$ is not satisfied by $s$ since
      $\not \models_{\fA} Px[s(x \mid q)]$ because $q \notin P^\fA$.
      Likewise, $\forall x \, Qx$ is not satisfied by $s$ since
      $\not \models_{\fA} Qx[s(x \mid p)]$ because $p \notin Q^\fA$.\\
      Therefore, $\forall x \, Px \lor \forall x \, Qx$ is not satisfied,
      so $\phi \not\models \psi$.
    \end{Answer}
  \item $\phi$ is $\forall x \, Px \lor \forall x \, Qx$,
    and $\psi$ is $\forall x \, (Px \lor Qx)$.
    \begin{Answer}
      In this case, $\phi \deduces \psi$.

      \step
      Let $\fA$ be a structure and $s$ a variable assignment such that
      $\models_{\fA} (\forall x\, Px \lor \forall x\, Qx)[s]$.\\
      Since $\sbar(\forall x\, Px \lor \forall x\, Qx)
      = \sbar(\forall x\, Px) \lor \sbar(\forall x\, Qx)$,
      either $\models_\fA \forall x\, Px[s]$ or $\models_\fA \forall x\, Qx[s]$.

      \begin{enumroman}
        \item When $\models_\fA \forall x\, Px[s]$ is satisfied,
          then $\models_\fA Px [s(x \mid d)]$ for every $d \in \abs{\fA}$.\\
          Therefore, $\models_\fA Px \lor Qx [s(x \mid d)]$ for every $d \in \abs{\fA}$,
          meaning $\forall x\, Px \lor Qx$ is also satisfied by $s$.
        \item When $\models_\fA \forall x\, Qx[s]$ is satisfied,
          then $\models_\fA Qx [s(x \mid d)]$ for every $d \in \abs{\fA}$.\\
          Therefore, $\models_\fA Px \lor Qx [s(x \mid d)]$ for every $d \in \abs{\fA}$,
          meaning $\forall x\, Px \lor Qx$ is also satisfied by $s$.
      \end{enumroman}
      in the first case, $\models_\fA \forall x\, Px[s]$, so
      $\fA$ satisfies $\forall x\, Px$.
    \end{Answer}
  \item $\phi$ is $\forall x \, (Px \lor Qy)$,
    and $\psi$ is $\forall x\, Px \lor \forall x \, Qy$.
    \begin{Answer}
      In this case, $\phi \deduces \psi$.

      \step
      % For example, take $\fA$ as a structure and $\abs{\fA} = \set{p, q}$
      % as a set of sentences.
      % Further, suppose $P^\fA = \set{p}$ and $Q^\fA = \set{q}$.

      \step
      Since $y$ occurs free, $\forall x \, (Px \lor Qy)$ is satisfied by $s$ iff
      $\models_{\fA} \forall x Px [s]$ or $\models_{\fA} Qy [s]$
      (by the generalization theorem, if $\fA \models\forall x Qy$
      then $\fA \models Qy$).

      \step
      In the first case, $\models_{\fA} \forall x Px$ deduces that that
      $\forall x Px$ is satisfied by $s$, which
      tautologically implies that $\forall x\, (Px \lor Qy)$ is satisfied by $s$.

      \step
      In the second case, $\models_{\fA} Qy$ deduces that that
      $\forall x\, (Px \lor Qy)$ is satisfied by $s$.

    \end{Answer}
\end{enumroman}

\end{problem}



\begin{problem}
  Show carefully and formally, directly from the formal definition of satisfaction
  \footnote{
    I repeat, directly from the formal, recursive definition of satisfaction,
    given on the bottom of page 83 and the top of page 84 in the textbook.
    In particular, this problem  is about logical validity, and not about deductions.
  },
  that the sentence \[\forall x\,  Pxfc \rightarrow \exists x Pxfx \]
  (where $c$ is a constant symbol, $f$ is a one-place function symbol,
  and $P$ is a two-place predicate symbol) is logically valid.
\end{problem}
\begin{Answer}
  \begin{blockcolor}
    \textbf{Formal definition of satisfaction:}

    \step
    $\models_\fA \phi \; [s]$ if and only if the translation of $\phi$
    determined by $\fA$ is true.

    \begin{itemize}
      \item Terms.
        \begin{enumroman}
          \item For each variable $x$, $\sbar(x) = s(x)$.
          \item For each constant $c$, $\sbar(c) = c^\fA$.
          \item If $t_1, \ldots, t_n$ are terms and $f$ is an $n$-place
            function symbol, then $\sbar(f t_1 \cdots t_n) = f^\fA(\sbar(t_1), \ldots, \sbar(t_n))$.
        \end{enumroman}
      \item Atomic formulas.
        \begin{enumroman}
          \item $\models_\fA = t_1 t_2$ iff $\sbar(t_1) = \sbar(t_2)$.
          \item For an $n$-place predicate parameter $P$,
            $\models_\fA P t_1 \cdots t_n$ iff
            $\vector{\sbar(t_1), \ldots, \sbar(t_n)} \in P^\fA$.
        \end{enumroman}
      \item Other wffs
        \begin{enumroman}
          \item Atomic formulas as defined above.
          \item $\models_\fA \lnot \phi \; [s]$ iff $\not \models_\fA \phi \; [s]$.
          \item $\models_\fA (\phi \lto \psi)\; [s]$ iff either
            $\not \models_\fA \phi [s]$ or $\models_\fA \psi [s]$,
              or both.
          \item $\models_\fA \forall x \; \phi \; [s]$ iff for every
            $d \in \abs{\fA}$, we have
            $\models_\fA \phi \; [s(x \mid d)]$.
        \end{enumroman}
    \end{itemize}
  \end{blockcolor}

  \bigskip

  \step
  Let $\fA$ be a structure and $s$ a variable assignment.\\
  Let $\gamma_1 = \forall x\,  Pxfc$ and $\gamma_2 = \exists x Pxfx$,
  such that the given sentence is equivalent to
  $(\gamma_1 \lto \gamma_2)$.
  Then the sentence is valid iff either $\gamma_1$ is not valid,
  or $\gamma_2$ is valid (or both).

  \step
  When $\gamma_1$ is not valid, it follows from the definition of the
  ``$\lto$'' connective that $(\gamma_1 \lto \gamma_2)$ is valid.
  
  \step
  Suppose $\gamma_1$ is valid, i.e. $\forall x\,  Pxfc = T$,
  we show that $\gamma_2$ must also be valid.\\
  By definition of satisfaction, $\forall x\,  Pxfc$ is valid iff
  for every $d \in \abs{\fA}$, we have $\models_\fA Pxfc \; [s(x \mid d)]$.\\
  This means that $\sbar(Pxfc) = P^\fA s(x) f^\fA c^\fA = T$ whenever
  $s(x) = d$ for some $d \in \abs{\fA}$.
  By definition of satisfaction, $\sbar(c) = c^\fA$, the translation of $c$
  into an element in $\abs{\fA}$.
  When we map $x$ to $c^\fA$, we get that $(Pxfc)^x_{c^{\fA}}$ is valid
  (since $(P x fc)^x_d$ is valid for all $d \in \abs{\fA}$).
  This evaluates to $P^\fA c^\fA f^\fA c^\fA = \sbar(Pcfc)$,
  so $Pcfc$ must also be valid, and
  $\exists x Pxfx$ is valid, particularly when $x = c$.
\end{Answer}

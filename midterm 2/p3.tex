\begin{problem}
  You are free to use any of the results in the textbook,
  including the Soundness, Completeness, and Compactness Theorems.

  \bigskip
  
  
  Let $\cL$ be a reasonable
  \footnote{
    This means reasonable in Enderton's sense;
    see page 142.}
  language for first-order logic, and let $T_1$ and $T_2$ be two theories of
  $\calL$.
  \begin{enumalph}
    \item Show that $T_1 \cap T_2$ is also a theory.\\
      (Recall that a set of sentences $T$ is a theory iff,
      for every sentence $\sigma$, we have
      $(T \deduces \sigma \;\implies\; \sigma \in T)$.)
    
    \begin{Answer}
      Let $\sigma$ be a sentence such that $T_1 \cap T_2 \deduces \sigma$,
      for $T_1 \cap T_2$ to be a theory, then $\sigma \in T_1 \cap T_2$.
      
      If $T_1 \cap T_2 \deduces \sigma$, then $T_1 \deduces \sigma$
      (since $T_1 \cap T_2 \subseteq T_1$) and $T_2 \deduces \sigma$
      (since $T_1 \cap T_2 \subseteq T_2$).
      However, if $T_1 \deduces \sigma$, then $\sigma \in T_1$ (since $T_1$
      is a theory), and if $T_2 \deduces \sigma$, then $\sigma \in T_2$
      (since $T_2$ is a theory). It follows that $\sigma \in T_1 \cap T_2$.
      
      Therefore, $T_1 \cap T_2 \deduces \sigma$, iff
      $\sigma \in T_1 \cap T_2$, so $T_1 \cap T_2$ is a theory.
    \end{Answer}

    \item Suppose that $T_1$ and $T_2$ are both 
      axiomatizable, complete, consistent theories of $\mathcal L$.
      Which of the following must be true?
      
      \step
      If true, explain why; if false, give a counterexample.

      \begin{enumroman}
        \item $T_1 \cap T_2$ is consistent.
          \begin{Answer}
            True.

            Since $T_1$ is a theory for $\calL$, every sentence in $T_1$
            is true in $\calL$. Therefore, every sentence in $T_1 \cap T_2$
            must be true in $\calL$ since $(T_1 \cap T_2) \subseteq T_1$.
            $T_2$ is also a theory for $\calL$, so every sentence in $T_2$
            is true in $\calL$ and the same argument can be made using $T_2$.

            Therefore, every deduction from $T_1 \cap T_2$ is also deduced
            by $T_1$ and $T_2$. Since $T_1$ and $T_2$ are consistent,
            $T_1 \cap T_2$ must also consistent.

            \step
            If $T_1 \cap T_2$ were inconsistent, then it would deduce
            --- hence contain --- some sentence $\sigma$ and its negation,
            $\lnot \sigma$, which would mean $T_1$ and $T_2$ would both
            contain $\sigma$ and $\lnot \sigma$, thus contradicting
            the consistency of $T_1$ and $T_2$.
          \end{Answer}
        \item $T_1 \cap T_2$ is complete.
          \begin{Answer}
            True.

            Since $T_1$ is complete, for every sentence $\sigma$,
            either $\sigma \in T_1$ or $(\lnot \sigma) \in T_1$.
            Likewise, since $T_2$ is complete, for every sentence $\sigma$,
            either $\sigma \in T_2$ or $(\lnot \sigma) \in T_2$.

            Since both $T_1$ and $T_2$ are theories for $\cL$,
            we cannot have $\sigma \in T_1$ and $(\lnot \sigma) \in T_2$
            (since all sentences in each of the theories have to be true in $\cL$,
            and both theories are consistent).
            Therefore, if $\sigma$ is a valid sentence in $\cL$,
            it must be in $T_1$ \emph{and} it must be in $T_2$,
            so it must be in $T_1 \cap T_2$.
            On the other hand, if $\sigma$ is not a valid sentence in $\cL$,
            then $\lnot \sigma$ must be in $T_1$ \emph{and}
            $\lnot \sigma$ must be in $T_2$, so $\lnot \sigma$ must be in
            $T_1 \cap T_2$.

            Therefore, $T_1 \cap T_2$ is also complete.
          \end{Answer}
        \newpage
        \item $T_1 \cap T_2$ is decidable.
          \begin{Answer}
            False.

            That the language $\cL$ is reasonable means we can effectively enumerate
            the set of valid sentences in $\cL$.
            However, this set is not necessarily decidable
            so, given any formula $\gamma$, we may not readily determine
            if it is valid or not valid in $\cL$. Therefore,
            we may not be able to determine if $\gamma \in T_1 \cap T_2$
            or $\lnot \gamma \in T_1 \cap T_2$,
            so $T_1 \cap T_2$ may not be decidable.
          \end{Answer}
      \end{enumroman}
  \end{enumalph}
\end{problem}
